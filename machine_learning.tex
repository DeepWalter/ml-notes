\documentclass[111pt, a4paper]{book}
\usepackage{amsmath, amssymb, amsbsy}
\usepackage[amsmath, thmmarks]{ntheorem}
\usepackage{algorithm}
\usepackage{algpseudocode}
\usepackage{xcolor}
\usepackage{tikz}
\usepackage{paralist}
\usepackage{nicefrac}
\usepackage{minted}
\usepackage[%
    bookmarksnumbered,%
    bookmarksopen,%
    pdftitle={Machine Learning Notes},%
    pdfauthor=DeepWalter,%
    colorlinks=true,%
    pagebackref=true
]{hyperref}
\usepackage[adjust, nocompress, space]{cite}
\usepackage{setspace}


% ============================================================================================================
% ============================================================================================================

% ============================= Spacing ========================================
% ========== Row Spacing ============
% Packages: setspace
\setstretch{1.2} % with extra 20% row spacing.


% ============================= Pseudocode Customization =======================
% Packages: algorithm, algpseudocode.
\algrenewcommand{\algorithmicrequire}{\textbf{Input:}}
\algrenewcommand{\algorithmicensure}{\textbf{Output:}}
\algnewcommand\algorithmicto{\textbf{to} } % a trailing space is needed.
\algnewcommand\algorithmicbreak{\textbf{break}} % break keyword.
\newcommand{\breakif}[1]{\State \textbf{if} {#1} \textbf{then break}}


% ============================= Math Theorem Envs ==============================
\theorembodyfont{\normalfont}
\newtheorem{df}{Definition}[chapter]
\newtheorem{thm}[df]{Theorem}
\newtheorem{eg}[df]{Example}
\newtheorem{prop}[df]{Proposition}
\newtheorem{lem}[df]{Lemma}
\newtheorem{cor}[df]{Corollary}
\newtheorem{re}[df]{Remark}

% Customize proof env.
\theoremstyle{nonumberplain} % no numbering
\theoremsymbol{$\square$}    % proof end with square.
\newtheorem{pf}{Proof}


% =============================== Operators ====================================
\DeclareMathOperator{\ent}{Entropy}
\DeclareMathOperator*{\p}{\mathbb{P}} % probability.
\DeclareMathOperator{\sign}{sign}
\DeclareMathOperator*{\argmax}{argmax}
\DeclareMathOperator*{\argmin}{argmin}
\DeclareMathOperator*{\E}{\mathbb{E}} % Expectation.
\DeclareMathOperator{\dist}{dist} % distance.
\DeclareMathOperator{\indi}{\mathbb{I}}
\DeclareMathOperator{\tr}{tr} % trace of an matrix.



% =============================== New Commands =================================
% Logical connectives:
\newcommand{\OR}{\textbf{OR} } % a space is needed here.
\newcommand{\AND}{\textbf{AND} } 
\newcommand{\XOR}{\textbf{XOR} }


% Math commands:
\newcommand{\T}[1]{\ensuremath{{#1}^\mathsf{T}}} % Matrix transposition.
\newcommand{\inv}[1]{\ensuremath{{#1}^{-1}}} % Inversion.
\newcommand{\V}[1]{\ensuremath{\boldsymbol{#1}}}
\newcommand{\forany}{\forall~} % append a non-breakable whitespace.
\newcommand{\thereis}{\exists~} 
\newcommand{\linspan}[1]{\ensuremath{\operatorname{\mathbf{span}}\left\{{#1}\right\}}}

% Shorthand commands:
\newcommand{\hypo}[1]{\ensuremath{{#1}: \mathcal{X} \longrightarrow \mathcal{Y}}} % Hypothesis function.
\newcommand{\dataset}{\ensuremath{D = \{(\V{x}_1, y_1), \dotsc, (\V{x}_m, y_m)\}}} % Labeled training set.
\newcommand{\st}{\text{s.t.\ }}
\newcommand{\magenta}[1]{\textcolor{magenta}{#1}} % turn the color of text into magenta.
\newcommand{\hl}[2][yellow]{\colorbox{#1}{#2}} % highlight the text.
\newcommand{\pfrac}[2]{\ensuremath{\frac{\partial {#1}}{\partial {#2}}}}
\newcommand{\angpair}[2]{\ensuremath{\langle {#1}, {#2}\rangle}} % inner product with angles.


% ============================= Python Code Customization ======================
% Package: minted
\setminted[python]{frame=single, linenos=true, numbersep=3pt, autogobble=true, mathescape=true}


% ========================================================================================================
% ============================================= Document =================================================
% ========================================================================================================
\begin{document}

% =====================Title Page====================
\title{\textbf{Machine Learning Notes}}
\author{DeepWalter\thanks{Email: deepwalter.cn@gmail.com}}
\date{}
\maketitle

\frontmatter
% =====================Table of Contents=============
\tableofcontents

% ===================================================
\mainmatter
% =============================Part One: Basics========================
\part{Basics}
% ========Model Assessment and Selection======
\chapter{Model Assessment and Selection}
\section{Model Assessment and Selection}

\subsection{Empirical Error and Overfitting}

\subsection{Assessment}

\subsection{Performance Measurement}

\subsection{Bias and Variance}

% ========Linear Models================
\chapter{Linear Model}
In a linear model, the learner takes the form of a (affine) linear function
$$y = \langle\V{w}, \V{x}\rangle + b = \langle\hat{\V{w}}, \hat{\V{x}}\rangle$$
where $\hat{\V{w}}=(\V{w}, b),\;\hat{\V{x}} = (\V{x}, 1)$.
\section{Linear Regression}

\section{Logistic Regression}

\section{Linear Discriminant Analysis}
The idea of linear discriminant analysis is to find a line with direction $\V{w}$ \st when
projected to this line, examples with the same label will stay close while examples with different
labels will be far away. When predicting, we first project the example onto the line, then assign it the
label of the group of examples which is closest to it.

Let \dataset\ be the data set, $\mathcal{Y} = \{0, 1\}$ the label space. Let\marginnote{$n_0 + n_1 = m$ and
$\V{\mu}_i$ is the center of examples with label $i$.} 
$X_i \in \mathbb{R}^{n_i \times n}$ be the matrix of all examples with label $i$ and 
$\V{\mu}_i \in \mathbb{R}^{1 \times n}$ the mean vector of all such examples. Let 
$$ \Sigma_i = \T{(X_i - \V{\mu}_i)}(X_i - \V{\mu}_i)$$
be the covariance matrix of all examples with label $i$. After
projecting onto the line $\V{w}$, $X_i$ becomes $X_i \T{\V{w}}$, and $\V{\mu}_i$ becomes
$\V{\mu}_i\T{\V{w}}$. Hence the covariance matrix becomes $\V{w} \Sigma_i \T{\V{w}}$, which is a real number.
To make examples with the same label stay close and examples with different labels stay away, we want small 
variances $\V{w} \Sigma_i \T{\V{w}}$ and a large distance $\norm{\V{\mu}_0\T{\V{w}} - \V{\mu}_1\T{\V{w}}}^2$.
That is, we want to maximize
\begin{equation}\label{lda_J}
    J = \frac{\norm{\V{\mu}_0\T{\V{w}} - \V{\mu}_1\T{\V{w}}}^2}{\V{w} \Sigma_0 \T{\V{w}} + 
    \V{w} \Sigma_1 \T{\V{w}}}
\end{equation}
If we define the \textbf{within-class scatter matrix} as
\begin{equation}
    S_w = \sum_{\V{x}\in X_0} \T{(\V{x} - \V{\mu}_0)}(\V{x} - \V{\mu}_0)
    + \sum_{\V{x}\in X_1} \T{(\V{x} - \V{\mu}_1)}(\V{x} - \V{\mu}_1)
\end{equation}
and \textbf{between-class scatter matrix} as
\begin{equation}
    S_b = \T{(\V{\mu}_0 - \V{\mu}_1)}(\V{\mu}_0 - \V{\mu}_1)
\end{equation}
then equation~\eqref{lda_J} becomes
\begin{equation}
    J = \frac{\V{w}S_b\T{\V{w}}}{\V{w}S_w\T{\V{w}}}
\end{equation}
which is the \textbf{generalized Rayleigh quotient} of $S_b$ and $S_w$.

To maximize the generalized Rayleigh quotient, we solve the equivalent problem
$$ \argmax_{\V{w}} \V{w} S_b \T{\V{w}}\quad\st \V{w}S_w \T{\V{w}} = 1$$
Let $f(\V{w}, \lambda) = \V{w} S_b \T{\V{w}} + \lambda(1 - \V{w}S_w \T{\V{w}})$ be the Lagrangian, then 
$\pfrac{f}{\V{w}} = 0$ implies
\begin{equation}
    \V{w} S_b = \lambda \V{w} S_w
\end{equation}
Since $\V{w} S_b = \alpha (\V{\mu}_0 - \V{\mu}_1)$ for some $\alpha$, we have
\begin{equation}\label{lda_w}
    \V{w} = \frac{\alpha}{\lambda} (\V{\mu}_0 - \V{\mu}_1)S_w^{-1}
\end{equation}
Notice that we have the restriction $\V{w}S_w\T{\V{w}} = 1$, which gives
\begin{equation}\label{lda_w_coeffi}
    \frac{\lambda}{\alpha} = \sqrt{(\V{\mu}_0 - \V{\mu}_1)S_w^{-1}\T{(\V{\mu}_0 - \V{\mu}_1)}}
\end{equation}
Combine equation~\eqref{lda_w} and~\eqref{lda_w_coeffi}, we have
\begin{equation}
    \V{w} = 
    \frac{(\V{\mu}_0 - \V{\mu}_1)S_w^{-1}}{\sqrt{(\V{\mu}_0 - \V{\mu}_1)S_w^{-1}\T{(\V{\mu}_0 - \V{\mu}_1)}}}
\end{equation}


% ========Decision Trees===============
\chapter{Decision Tree}
\section{Classification Tree}
In this section, we focus on classification trees. First we assume that the features are discrete. We want to
find a tree that can determine to which class an example belongs \magenta{given any combination of features}.
\subsection{General Algorithm}
The general algorithm for generating a classification tree is very simple. It recursively splits the
node into subtrees according to a given rule until some stop conditions occur.
% Pseudocode of decision tree algorithm.
\begin{algorithm}
    \caption{Decision Tree}\label{decision_tree}
    \begin{algorithmic}[1]
        \Require training set $D = \{(x_1, y_1), (x_2, y_2), \ldots, (x_m, y_m)
        \}$; attribute set $A = \{a_1, a_2, \ldots, a_d\}$.
        \Ensure a decision tree.
        \Procedure{DT}{$D, A$}
            \State Generate a node.
            \If{$y_i = c, \forall~ i = 1, \ldots, m$} \Comment{All samples have 
            the same label}
                \State mark the node as a leaf with label $c$.
                \State \Return
            \EndIf
            \If{$A = \emptyset$ \OR samples of $D$ take the same value on each
            attribute from $A$}\Comment{attributes from $A$ cannot distinguish
            elements of $D$}
                \State mark the node as a leaf with the majority label of $D$. 
                \State \Return
            \EndIf
            \State Choose the \textit{best} attribute $a_*$ from $A$.\label{measurement}
            \If{\NOT worthSplitting$(D, a_*)$}\label{worthSplitting}\Comment{Not worth splitting.}
                \State mark the node as a leaf with the majority label of $D$.
                \State \Return
            \Else
                \ForAll{possible value $a_*^v$ of $a_*$}
                    \State generate a branch from the node.
                    \State let $D_v$ be all samples that take value $a_*^v$ on attribute
                    $a_*$.
                    \If{$D_v = \emptyset$} \Comment{no sample of $D$ takes value $a_*^v$
                    on $a_*$}
                        \State mark the branch as a leaf with the majority label of 
                        $D$.
                        \State \Return
                    \Else
                        \State set the branch to \Call{DT}{$D_v, A-\{a_*\}$}.
                    \EndIf
                \EndFor
            \EndIf
        \EndProcedure
    \end{algorithmic}
\end{algorithm}

\subsection{Measurement}

The line~\algref{decision_tree}{measurement} and~\algref{decision_tree}{worthSplitting} are the key points of
the algorithm. Namely, we need some reasonable measurement to choose the best feature and decide whether it is
necessary to do the split. As ususal, we find some loss function $L$ which is defined on a single node.
For any feature $a$, let ${\{D^v\}}_v$ be a splitting of $D$ according to it. Then $L$ can be extended to 
${\{D^v\}}_v$, the tree after splitting, in a natural way:
\begin{equation}\label{loss_tree}
    L(D, a) = \sum_v \frac{|D^v|}{|D|}L(D^v)
\end{equation}
Our goal is to find a feature that minimize the above loss. That is, the best feature $a_*$ is given by:
$$a_* = \argmin_a L(D, a)$$
Notice that before splitting, the loss on the single node is $L(D)$. Hence we can define the gain of a
splitting\footnote{i.e.\ the amount of loss reduced by the splitting.} as 
\begin{equation}
    \gain(D, a) = L(D) - L(D, a)
\end{equation}
If the gain is too small, for example, less than a preset threshold $\varepsilon \geqslant 0$, then we decide
that it is not worth splitting. And also notice that 
$$\argmax_a \gain(D, a) = \argmin_a L(D, a)$$
We can combine the above two metrics into a single one.

\subsubsection{Information Gain}
\begin{df}[Information Entropy]
    The \textbf{information entropy} of a discrete random variable $X$ is defined as:
    \begin{equation}
        H(X) = \E[-\log_2(\p(X))]
    \end{equation}
    If $X \in \{x_1, \dotsc, x_n\}$ and $\p(x_i) = p_i$\,, then the above entropy can be written as:
    $$H(X) = -\sum_{i=1}^n p_i\log_2 p_i$$
\end{df}
\begin{re}
    It is easy to see that:
    \begin{compactenum}
        \item $H(X) \geqslant 0$
        \item $H$ takes the minimal value when $p_1 = 1$. 
    \end{compactenum}
\end{re}

For any training dataset \dataset, let $\mathcal{Y}$ be the label space and $Y \in \mathcal{Y}$ the random
variable determined by $D$.
We can define the entropy of $D$ as the entropy of $Y$:
\begin{equation}
    \ent(D) = - \sum_{i=1}^{|\mathcal{Y}|}p_i\log_2 p_i
\end{equation}
where $p_i$ is the probability of the $i$-th label occuring in $D$.

Similarly, we have the conditional entropy of two random variables $X, Y$:
\begin{df}[Conditional Entropy]
    The entropy of $X$ given $Y$ is defined as
    \begin{equation}
        \begin{split}
        H(X | Y) &= \sum_{y} p(y) H(X | Y=y)\\
                 &= - \sum_{y}p(y) \sum_{x} p(X=x | Y=y)\log_2 p(X=x | Y=y)\\
                 &= -\sum_{x, y} p(x, y)\log_2 \frac{p(x, y)}{p(y)}
        \end{split}
    \end{equation}
\end{df}

Notice that the conditional entropy has the same form as the loss in~\eqref{loss_tree}.
That is, for any possible value $a^v$ of $a$, let
$D^v = \{\V{x}_i | y_i = a^v\}$, i.e.\ $D^v$ is the collection of those example with label $a^v$. Then the
conditional entropy of $D$ w.r.t.\ $a$ can be written as
\begin{equation}
    \ent(D, a) = \sum_{v} \frac{|D^v|}{|D|}\ent(D^v)
\end{equation}
Hence the \textbf{information gain} of the splitting according to $a$ is\marginnote{Remember we want to 
reduce\\ the entropy of the tree.}
\begin{equation}
    \begin{split}
    \gain(D, a) &= \ent(D) - \ent(D, a)\\
                &= \ent(D) - \sum_{v} \frac{|D^v|}{|D|}\ent(D^v)
    \end{split}
\end{equation}
Of course, we want to choose $\displaystyle a_* = \max_a \gain(D, a)$ as the best feature in 
line~\algref{decision_tree}{measurement}. And it is worth splitting the tree only if 
$\gain(D, a_*) \geqslant \varepsilon$
for some preset threshold $\varepsilon \geqslant 0$. This kind of measurement is adopted by the ID3 algorithm.

\subsubsection{Information Gain Ratio}
When all samples in $D$ have the same label, then it's easy to see that
$$\ent(D) = - 1 \cdot \log_2 1 = 0$$
Hence if some feature takes different values on each example, then the conditional entropy is $0$, and the
information gain is maximal. That is, information gain favours those features with more possible values. To 
reduce this bias, we introduce the information gain ratio. Let 
$$\operatorname{IV}(a) = - \sum_v \frac{|D^v|}{|D|}\log_2\frac{|D^v|}{|D|}$$
be the \textbf{intrinsic value} of $a$, then the information gain ratio w.r.t.\ $a$ is defined as:
\begin{equation}
    \operatorname{GainRatio}(D, a) = \frac{\gain(D, a)}{\operatorname{IV}(a)}
\end{equation}
Notice that when $a$ only takes one value, then $\operatorname{IV}(a) = 0$. Hence the information gain ratio
favours those features with less values. In order to balance between gain and gain ratio, we can 
\magenta{first choose
those features with information gain above average, then choose the one with highest gain ratio from them}, as
in C4.5 algorithm. As ususal, we can compare the gain ratio with the threshold $\varepsilon$ to determine
whether to split the tree or not.

\subsubsection{Gini Index}
The Gini index of the training set $D$ is
\begin{equation}
    \begin{split}
    \operatorname{Gini}(D) &= \sum_{i}\sum_{i'}p_i p_{i'}\\
                           &= 1 - \sum_{i=1}^{|\mathcal{Y}|} p_i^2
    \end{split}
\end{equation}
Namely, it is the probability that you get two different labels when you randomly pick two examples from $D$.
That is, it reflects the purity of the node. As usual, the Gini index of $D$ w.r.t.\ $a$ can be defined as:
\begin{equation}
    \operatorname{Gini}(D, a) = \sum_v \frac{|D^v|}{|D|}\operatorname{Gini}(D^v)
\end{equation}
which measures the purity of the tree after splitting. Then we can use Gini index as the loss function in the
measurement.


\subsection{Pruning}

\section{Classification and Regression Tree}

% ========SVM========================
\chapter{Support Vector Machine}
Let \dataset\ be the training set where $y_i \in\{-1, +1\}$. The SVM is an algorithm that tries to find a 
hyperplane $\langle\V{w}, \V{x}\rangle + b = 0$ which separate the positive examples from the negative ones. The 
corresponding predictor is $f(\V{x}) = \sign(\langle\V{w}, \V{x}\rangle + b)$.

\section{Hard SVM}
In the case of hard SVM, we assume that the training set is \magenta{linear separable}. Before going any 
further about the ideas of SVM, let's first introduce the concept of the margin of a hyperplane:
% margin of a hyperplane w.r.t. a training set.
\begin{df}[Margin]
    The margin of a hyperplane $\langle\V{w}, \V{x}\rangle + b = 0$ w.r.t.\ a training set  $D$ is defined as
    \begin{equation*}
    \min_i \frac{y_i(\langle\V{w}, \V{x}_i\rangle + b)}{||\V{w}||}
    \end{equation*}
    Obviously, positive margin means the hyperplane classifies $D$ correctly while negative margin means
    there is at least one point on the wrong side.
    When $D$ is correctly classified by the hyperplane, the margin is just the geometric distance between the
    training set and the hyperplane.
\end{df}

% analysis of the hard SVM algorithm
The core idea of (hard) SVM is to find a hyperplane which separates the training set
\textit{with the largest margin}. That is, we hope to solve the following:\marginnote{Obviously, for any 
positive $\lambda$, $(\V{w}, b)$ and $(\lambda\V{w}, \lambda b)$ specify the same hyperplane with the same
``left'' and ``right'' sides.}
\begin{equation}\label{SVM_original}
    \argmax_{\V{w}, b}\min_i \frac{y_i(\langle\V{w}, \V{x}_i\rangle + b)}{||\V{w}||}
\end{equation}
Let
$$\gamma =\min_i \frac{y_i(\langle\V{w}, \V{x}_i\rangle + b)}{||\V{w}||}$$
then the problem~\eqref{SVM_original} becomes:
\begin{equation}
    \argmax_{\V{w}, b} \gamma\;,\quad \st \frac{y_i(\langle\V{w}, \V{x}_i\rangle + b)}{||\V{w}||} \geqslant
     \gamma \quad\forall\ i
\end{equation}
Let $\gamma \gets ||\V{w}||\gamma$, the above problem becomes:
\begin{equation}
    \argmax_{\V{w}, b} \frac{\gamma}{||\V{w}||}\;,\quad \st y_i(\langle\V{w}, \V{x}_i\rangle + b) \geqslant
    \gamma \quad\forall\ i
\end{equation}
Using the linear separable assumption, we know that $\gamma > 0$. Let $\V{w} \gets \dfrac{\V{w}}{\gamma}$
and $b \gets \dfrac{b}{\gamma}$, then the above becomes:
\begin{equation}\label{SVM_argmax}
    \argmax_{\V{w}, b} \frac{1}{||\V{w}||}\;,\quad \st y_i(\langle\V{w}, \V{x}_i\rangle + b) \geqslant 1
    \quad\forall\ i
\end{equation}
which is obviously equivalent to:
\begin{equation}\label{hard_SVM}
    \argmin_{\V{w}, b} \frac{||\V{w}||^2}{2}\;,\quad \st y_i(\langle\V{w}, \V{x}_i\rangle + b) \geqslant 1
    \quad\forall\ i
\end{equation}

% validity of the above analysis.
\begin{thm}
%\begin{sloppypar}
    Assume the training set \dataset\ is \magenta{linear separable}, then there exists a unique hyperplane 
    separating the dataset with the largest margin, which is given by the solution of the above 
    problem~\eqref{hard_SVM}.
%\end{sloppypar}
\end{thm}
\begin{pf}
    The existence part is immediately from the separability assumption.\\
    TODO
\end{pf}

\begin{re}
    From the problem~\eqref{SVM_argmax}, it is clear that if $(\V{w}, b)$ is the separating hyperplane, then
    $\exists~i\ \st y_i(\langle\V{w}, \V{x}_i\rangle + b) = 1$ and the (largest) margin is $\frac{1}{||\V{w}||}$.
    Since the hyperplane is totally determined\footnote{that is, you can remove other points without affecting
    the separating hyperplane.} by those $\V{x}_i$\,s, they are called the \textbf{supporting vectors} of the
    hyperplane.
\end{re}

\section{Soft SVM}
The hard SVM works well when the data set is linear separable, but it behaves poorly on the sets which are not
linear separable since all the restrictions cannot be satisfied at the same time. In order to adapt to the 
non-separable case, we can allow some points to break the restriction, and penalize them in the optimization
target. That is, we can consider the following problem:
\begin{equation}\label{soft_SVM_original}
    \argmin_{\V{w}, b} \frac{||\V{w}||^2}{2} + C\sum_{i=1}^{m}\max\big(0, 1 - y_i(\langle\V{w}, \V{x}_i\rangle
    + b)\big)
\end{equation}
Here, if $y_i(\langle\V{w}, \V{x}_i\rangle + b) < 1$, we add the penalization 
$1 - y_i(\langle\V{w}, \V{x}_i\rangle +b)$ to the 
optimization target; otherwise, no penalization is added. The constant $C > 0$ is the weight that determines 
how much the penalization matters (or how much you can violate the restrictions). For example, if $C = 0$, 
then the penalization dosen't matter and you can violate all the restrictions; if $C = +\infty$, then the 
penalization matters most and you cannot violate any single restriction.\par
If we introduce the slack variabels $\xi_i = \max\big(0, 1 - y_i(\langle\V{w}, \V{x}_i\rangle +b)\big)$, then
the problem~\eqref{soft_SVM_original} becomes:
\begin{equation}\label{soft_SVM}
    \argmin_{\V{w}, b, \V{\xi}} \frac{||\V{w}||^2}{2} + C\sum_{i=1}^{m}\xi_i\quad\st\left\{
    \begin{aligned}
    & y_i(\langle\V{w}, \V{x}_i\rangle + b) \geqslant 1 - \xi_i\\
    & \xi_i \geqslant 0 
    \end{aligned}\right.
    \quad\forall~i
\end{equation}
this is what we called the soft SVM\@.

\section{Duality}

% duality of hard-SVM
Let $h_i(\V{w}, b) = 1 - y_i(\langle \V{w}, \V{x}_i\rangle + b)$, then the hard SVM~\eqref{hard_SVM} becomes
\begin{equation}\label{hard_SVM_reformatted}
\argmin_{\V{w},b} \frac{1}{2}||\V{w}||^2\quad\st h_i(\V{w}, b) \leqslant 0\quad\forall~i
\end{equation}
Its Lagrangian is
\begin{equation}\label{Lagrangian_hard_SVM}
    L(\V{w}, b; \V{\alpha}) = \frac{1}{2}||\V{w}||^2 + \sum_i \alpha_i h_i(\V{w}, b)
\end{equation}
The orignal problem~\eqref{hard_SVM_reformatted} above is equivalent to
$$\argmin_{\V{w}, b}\max_{\V{\alpha}: \alpha_i \geqslant 0} L(\V{w}, b; \V{\alpha})$$
Let $\displaystyle\theta_D(\V{\alpha}) = \min_{\V{w}, b}L(\V{w}, b; \V{\alpha})$, then 
$\nabla_{\V{w}, b} L(\V{w}, b; \V{\alpha}) = 0$ gives
\begin{subequations}
    \begin{align}
    &\V{w} = \sum_i \alpha_i y_i \V{x}_i\label{sub:w}\\
    &\sum_i y_i \alpha_i = 0\label{sub:b}
    \end{align}
\end{subequations}
Substitute equation~\eqref{sub:w} into the Lagrangian~\eqref{Lagrangian_hard_SVM}, we have
\begin{equation}
    \theta_D(\V{\alpha}) = \sum_i \alpha_i - \frac{1}{2}\sum_{i, j}\alpha_i\alpha_j y_i y_j\langle \V{x}_i, 
    \V{x}_j\rangle
\end{equation}
Hence the dual problem $\displaystyle \max_{\V{\alpha}: \alpha_i \geqslant 0} \argmin_{\V{w}, b}
L(\V{w}, b; \V{\alpha}) = \max_{\V{\alpha}: \alpha_i \geqslant 0}\theta_D(\V{\alpha})$ becomes
\begin{equation*}
    \max_{\V{\alpha}}\sum_i \alpha_i - \frac{1}{2}\sum_{i, j}\alpha_i\alpha_j y_i y_j\langle \V{x}_i, \V{x}_j
    \rangle\quad\st \left\{
    \begin{aligned}
    &\sum_i y_i \alpha_i = 0\\
    &\alpha_i \geqslant 0\quad\forall~i
    \end{aligned}\right.
\end{equation*}
or equivalently
\begin{equation}\label{dual_hard_SVM}
    \min_{\V{\alpha}}\frac{1}{2}\sum_{i, j}\alpha_i\alpha_j y_i y_j\langle \V{x}_i, \V{x}_j\rangle -
    \sum_i \alpha_i \quad\st \left\{
    \begin{aligned}
        &\sum_i y_i \alpha_i = 0\\
        &\alpha_i \geqslant 0\quad\forall~i
        \end{aligned}\right.
\end{equation}
Notice that $\frac{1}{2}||\V{w}||^2$ is convex and $h_i$ are affine linear. And the linear separable assumption
guarantees that there is some $(\V{w}, b)$ \st $h_i(\V{w}, b) < 0\;\forall~i$, i.e.\ the inequality 
restrictions are strict. Hence there is a solution
$(\V{w}^*, b^*)$ for the hard SVM~\eqref{hard_SVM}, and a solution $\V{\alpha}^*$ for the dual
problem~\eqref{dual_hard_SVM}. Moreover, they satisfy the KKT condition:
\begin{equation}\label{KKT_hard_SVM}
    \begin{cases}
        &\V{w}^* - \sum_i \alpha_i^* y_i \V{x}_i = 0\\
        &\sum_i y_i \alpha_i^* = 0\\
        & \alpha_i^* \geqslant 0\\
        & h_i(\V{w}^*, b^*) \leqslant 0\\
        & \alpha_i^* h_i(\V{w}^*, b^*) = 0
    \end{cases}
\end{equation}
Notice that not all $\alpha_i^*$ could be $0$ (otherwise $\V{w}^* = 0$, which is not a solution of the hard SVM).
Let $\alpha^*_{i_0} > 0$. Then $\alpha_{i_0}^* h_{i_0}(\V{w}^*, b^*) = 0$ implies
\begin{equation*}
    b^* = y_{i_0} - \sum_i \alpha_i^* y_i \langle \V{x}_i, \V{x}_{i_0}\rangle
\end{equation*}
In summary, we have
\begin{equation}\label{solution_hard_SVM}
    \begin{cases}
        &\V{w}^* = \sum_i \alpha_i^* y_i \V{x}_i\\
        &b^* = y_{i_0} - \sum_i \alpha_i^* y_i \langle \V{x}_i, \V{x}_{i_0}\rangle
    \end{cases}
\end{equation}
\begin{re}
    For those indices $i$ \st $\alpha^*_i > 0$, the corresponding $\V{x}_i$ are the supporting vectors since
    $h_i(\V{w}^*, b^*) = 0$. And it is easy to see that those $\V{x}_i$ with $\alpha^*_i = 0$ do not affect the
    hyperplane.
\end{re}

% duality of soft-SVM
% TODO: supply more details.
Similarly, let $h_i(\V{w}, b, \xi_i) = 1 - \xi_i - y_i(\langle\V{w}, \V{x}_i\rangle + b)$. Then the Lagrangian
of the soft SVM~\eqref{soft_SVM} is
$$L(\V{w}, b, \V{\xi}; \V{\alpha}, \V{\beta}) = \frac{1}{2}||\V{w}||^2 + C\sum_{i=1}^m \xi_i + \sum_{i=1}^m
\alpha_i h_i(\V{w}, b, \xi_i) + \sum_{i=1}^m \beta_i (-\xi_i)$$
The soft SVM~\eqref{soft_SVM} is equivalent to
$$\argmin_{\V{w}, b, \V{\xi}}\max_{\V{\alpha}, \V{\beta}: \alpha_i, \beta_i \geqslant 0}
L(\V{w}, b, \V{\xi}; \V{\alpha}, \V{\beta})$$
Let $\displaystyle\theta_D(\V{\alpha}, \V{\beta}) = \min_{\V{w}, b, \V{\xi}}L(\V{w}, b, \V{\xi}; \V{\alpha}, 
\V{\beta})$, then $\nabla_{\V{w}, b, \V{\xi}} L = 0$ implies
\begin{subequations}
    \begin{align}
    &\V{w} = \sum_i \alpha_i y_i \V{x}_i\label{sub:soft_w}\\
    &\sum_i y_i \alpha_i = 0\label{sub:soft_b}\\
    &\alpha_i + \beta_i = C\label{sub:soft_C}
    \end{align}
\end{subequations}
Hence we have:
\begin{equation}
    \theta_D(\V{\alpha}, \V{\beta}) =  \sum_i \alpha_i - \frac{1}{2}\sum_{i, j}\alpha_i\alpha_j y_i y_j\langle
    \V{x}_i, \V{x}_j\rangle
\end{equation}
which is exactly the same as before. Hence the dual problem of soft SVM~\eqref{soft_SVM} is
\begin{equation}\label{dual_soft_SVM}
    \min_{\V{\alpha}} \frac{1}{2}\sum_{i,j} \alpha_i \alpha_j y_i y_j\langle \V{x}_i, \V{x}_j\rangle -
    \sum_i \alpha_i\quad\st \left\{
    \begin{aligned}
        &\sum_i \alpha_i y_i = 0\\
        &0 \leqslant \alpha_i \leqslant C \quad\forall~i
    \end{aligned}\right.
\end{equation}
Let $\V{w}^*, b^*, \V{\xi}^*$ be the solution to the original soft SVM~\eqref{soft_SVM_original} and 
$\V{\alpha}^*, \V{\beta}^*$ the solution to the dual problem~\eqref{dual_soft_SVM}, then they satifies the KKT
conditions:
\begin{equation}\label{KKT_soft_SVM}
    \begin{cases}
        &\V{w}^* - \sum_i \alpha_i^* y_i \V{x}_i = 0\\
        &\sum_i y_i \alpha_i^* = 0\\
        & a_i^* + \beta_i^* = C\\
        & \alpha_i^* \geqslant 0\\
        & \beta_i^* \geqslant 0\\
        & \xi_i^* \geqslant 0\\
        & h_i(\V{w}^*, b^*, \xi_i^*) \leqslant 0\\
        & \alpha_i^* h_i(\V{w}^*, b^*, \xi_i^*) = 0\\
        & \beta_i^* \xi_i^* = 0
    \end{cases}
\end{equation}
If $\alpha_i^* > 0$, then $h_i(\V{w}^*, b^*, \xi_i^*) = 0$, i.e.\ 
$y_i(\langle\V{w}^*, \V{x}_i\rangle + b^*) = 1 - \xi^*_i$. Those $\V{x}_i$ are called supporting vectors. 
Moreover, if $\alpha^*_i < C$, then $\beta^*_i > 0$, hence $\xi^*_i = 0$ and
$y_i(\langle\V{w}^*, \V{x}_i\rangle + b^*) = 1$, thus those supporting vectors locate on the decision 
boundaries; if $\alpha^*_i = C$, then $\beta^*_i = 0$, thus those supporting vectors locate between the
decision boundaries if $\xi^*_i \leqslant 1$ and they are mis-classified if $\xi^*_i > 1$. If $\alpha^*_i = 0$,
then $\beta^*_i = C$ and $\xi^*_i = 0$, thus $1 \leqslant y_i(\langle\V{w}^*, \V{x}_i\rangle + b^*)$, that is 
they are correctly classified\footnote{but they don't affect the separating hyperplane.}.

If there is some $\alpha_{i_0}$ such that 
$0 < \alpha_{i_0}^* < C$, then the solution of the soft SVM~\eqref{soft_SVM} can be written as:
\begin{equation}
    \begin{cases}
        &\V{w}^* = \sum_i y_i \alpha^*_i \V{x}_i\\
        &b^* = y_{i_0} - \sum_i y_i \alpha^*_i \langle \V{x}_i, \V{x}_{i_0}\rangle
    \end{cases}
\end{equation}

\section{Kernel Method}
The general idea of kernel method is that after mapping the original feature space into an Hilbert space via
a map $\phi$, if we need to compute the inner product \angpair{\phi(\V{x}_i)}{\phi(\V{x}_j)} which usually is
difficult, we hope that there is a kernel function $\kappa$ \st $\angpair{\phi(\V{x}_i)}{\phi(\V{x}_j)} =
\kappa(\V{x}_i, \V{x}_j)$ and the latter is easier to compute. Following this idea, the dual problem~\eqref{dual_hard_SVM} in
Hilbert space
\begin{equation}
    \min_{\V{\alpha}}\frac{1}{2}\sum_{i, j}\alpha_i\alpha_j y_i y_j\langle \phi(\V{x}_i), \phi(\V{x}_j)\rangle
    - \sum_i \alpha_i \quad\st \left\{
    \begin{aligned}
        &\sum_i y_i \alpha_i = 0\\
        &\alpha_i \geqslant 0\quad\forall~i
        \end{aligned}\right.
\end{equation}
becomes
\begin{equation}
    \min_{\V{\alpha}}\frac{1}{2}\sum_{i, j}\alpha_i\alpha_j y_i y_j \kappa(\V{x}_i, \V{x}_j)
    - \sum_i \alpha_i \quad\st \left\{
    \begin{aligned}
        &\sum_i y_i \alpha_i = 0\\
        &\alpha_i \geqslant 0\quad\forall~i
        \end{aligned}\right.
\end{equation}
and its solution satifies
\begin{equation}
    \begin{cases}
        &\V{w}^* = \sum_i \alpha_i^* y_i \phi(\V{x}_i)\\
        &b^* = y_{i_0} - \sum_i \alpha_i^* y_i \kappa(\V{x}_i, \V{x}_{i_0})
    \end{cases}
\end{equation}
Hence the predictor is 
$$y = \sign\left(\sum_i \alpha_i^* y_i \kappa(\V{x}_i, \V{x}) + y_{i_0} - \sum_i \alpha_i^* y_i 
\kappa(\V{x}_i, \V{x}_{i_0})\right)$$

\begin{thm}[Kernel function]
    Let $\mathcal{X}$ be a feature space, $\kappa(\cdot\,, \cdot)$ is some symmetric function on 
    $\mathcal{X} \times \mathcal{X}$, then $\kappa$ is a kernel function if and only if for any data set
    $D = \{\V{x}_1, \dotsc, \V{x}_m\}$, the matrix
    \begin{equation*}
        \begin{bmatrix}
            \kappa(\V{x}_1, \V{x}_1) &\cdots &\kappa(\V{x}_1, \V{x}_j) &\cdots &\kappa(\V{x}_1, \V{x}_m)\\
            \vdots                   &\ddots &\vdots                   &\ddots &\vdots\\
            \kappa(\V{x}_i, \V{x}_1) &\cdots &\kappa(\V{x}_i, \V{x}_j) &\cdots &\kappa(\V{x}_i, \V{x}_m)\\
            \vdots                   &\ddots &\vdots                   &\ddots &\vdots\\
            \kappa(\V{x}_m, \V{x}_1) &\cdots &\kappa(\V{x}_m, \V{x}_j) &\cdots &\kappa(\V{x}_m, \V{x}_m)
        \end{bmatrix}
    \end{equation*}
    is semi-positive.
\end{thm}

\begin{prop}
    Let $\kappa_1$ and $\kappa_2$ be any kernel functions. Then
    \begin{compactenum}
        \item For any $\gamma_1, \gamma_2 > 0$, $\gamma_1 \kappa_1 + \gamma_2 \kappa_2$ is a kernel function.
        \item $\kappa_1 \cdot \kappa_2$ is a kernel function.
        \item For any function $g$, $\kappa(\V{x}, \V{y}) := g(\V{x})\kappa_1(\V{x}, \V{y})g(\V{y})$ is a 
        kernel function.
    \end{compactenum}
\end{prop}
Some useful kernels are:
\begin{compactenum}
    \item Linear kernel: $\kappa(\V{x}_1, \V{x}_2) = \langle \V{x}_1, \V{x}_2 \rangle$.
    \item Polynomial kernel: $\kappa(\V{x}_1, \V{x}_2) = {\langle \V{x}_1, \V{x}_2 \rangle}^d$ where 
    $d \geqslant 1$.
    \item Guassian kernel: $\kappa(\V{x}_1, \V{x}_2) = \exp(-\frac{||\V{x}_1 - \V{x}_2||^2}{2\sigma^2})$ where
    $\sigma > 0$.
    \item Laplacian kernel: $\kappa(\V{x}_1, \V{x}_2) = \exp(-\frac{||\V{x}_1 - \V{x}_2||}{\sigma})$ where
    $\sigma > 0$.
    \item Sigmoid kernel: $\kappa(\V{x}_1, \V{x}_2) = \tanh(\beta \langle \V{x}_1, \V{x}_2\rangle + \theta)$
    where $\beta > 0,\;\theta < 0$.
\end{compactenum}

\section{Support Vector Regression}
Although SVM is used for binary classification, we can apply its idea to linear regression. Namely, we can
tolerate those examples that are close enough to the predictor and only penalize those that are far away.
More specifically, we want to find the predictor $f(\V{x}) = \langle\V{w}, \V{x}\rangle + b$ by minimizing the loss
\marginnote{$L$ is just a regularized loss.}
\begin{equation}\label{SVR_loss}
    L(\V{w}, b) := \frac{1}{2}||\V{w}||^2 + C\sum_{i=1}^m l_\varepsilon(\hat{y}_i - y_i)
\end{equation}
where $l_\varepsilon$ is the $\varepsilon$-insensitive loss:
\begin{equation}\label{epsilon_insensitive_loss}
    l_\varepsilon(z) = 
    \begin{cases}
        0, & \text{if}~ |z| < \varepsilon\\
        |z| - \varepsilon, & \text{otherwise}
    \end{cases}
\end{equation}
Notice that $l_\varepsilon(z) = \max(|z| - \varepsilon, 0) = \max(z - \varepsilon, -z - \varepsilon, 0)$, if
we introduce the slack variabels $\xi_i = l_\varepsilon(\hat{y}_i - y_i)$, the problem becomes
\begin{equation}
    \argmin_{\V{w}, b, \V{\xi}} \frac{1}{2}||\V{w}||^2 + C\sum_{i=1}^m \xi_i \quad\st\quad\left\{
    \begin{aligned}
        &\xi_i \geqslant 0\\
        &\xi_i \geqslant \hat{y}_i - y_i - \varepsilon\\
        &\xi_i \geqslant -\hat{y}_i + y_i - \varepsilon
    \end{aligned}\right.\forany i
\end{equation}
Its Lagrangian is:
\begin{multline}
    L(\V{w}, b, \V{\xi}; \V{\alpha}, \V{\beta}, \V{\gamma}) = \frac{1}{2} ||\V{w}||^2 + C\sum_{i=1}^m \xi_i
    + \sum_{i=1}^m \alpha_i (-\xi_i) + \sum_{i=1}^m \beta_i (\hat{y}_i - y_i -\xi_i -\varepsilon)
    \\+ \sum_{i=1}^m \gamma_i (-\hat{y}_i + y_i - \xi_i -\varepsilon)
\end{multline}
Its dual problem is 
$$\max_{\substack{\V{\alpha}, \V{\beta}, \V{\gamma}\\ \alpha_i, \beta_i, \gamma_i \geqslant 0}}
\min_{\V{w}, b, \V{\xi}} L(\V{w}, b, \V{\xi};\V{\alpha}, \V{\beta}, \V{\gamma})$$
For the inner minimization problem,
$$\nabla_{\V{w}, b, \V{\xi}}L(\V{w}, b, \V{\xi};\V{\alpha}, \V{\beta}, \V{\gamma}) = 0$$
gives
\begin{subequations}
    \begin{align}
        &\V{w} = \sum_i (\gamma_i - \beta_i) \V{x}_i\\
        &\sum_i (\beta_i - \gamma_i) = 0\\
        &C = \alpha_i + \beta_i + \gamma_i
    \end{align}
\end{subequations}
Substitute those equations into the dual problem, we get
\begin{equation*}
    \max_{\substack{\V{\alpha}, \V{\beta}, \V{\gamma}\\ \alpha_i, \beta_i, \gamma_i \geqslant 0}} 
    -\frac{1}{2}\sum_{i, j}(\beta_i - \gamma_i)(\beta_j - \gamma_j)
    \langle\V{x}_i, \V{x}_j\rangle + \sum_i \big(y_i(\gamma_i - \beta_i) - \varepsilon(\gamma_i + \beta_i)\big)
\end{equation*}
which is equivalent to
\begin{equation}
    \min_{\substack{\V{\beta}, \V{\gamma} \\ \beta_i, \gamma_i \geqslant 0}} \frac{1}{2}
    \sum_{i, j}(\beta_i - \gamma_i)(\beta_j - \gamma_j)
    \langle\V{x}_i, \V{x}_j\rangle + \sum_i \big(\varepsilon(\beta_i + \gamma_i) + y_i(\beta_i - \gamma_i)\big)
\end{equation}
subjecting to the conditions
\begin{equation*}
    \begin{cases}
        & \sum_i (\beta_i - \gamma_i) = 0\\
        & \beta_i + \gamma_i \leqslant C\\
        & \beta_i \geqslant 0\\
        & \gamma_i \geqslant 0
    \end{cases}
\end{equation*}
Moreover, the KKT condition is
\begin{equation}
    \begin{cases}
        &\V{w} = \sum_i (\gamma_i - \beta_i) \V{x}_i\\
        &\sum_i (\beta_i - \gamma_i) = 0\\
        &C = \alpha_i + \beta_i + \gamma_i\\
        &\alpha_i, \beta_i, \gamma_i \geqslant 0\\
        &\xi_i \geqslant 0\\
        &\hat{y}_i - y_i - \xi_i - \varepsilon \leqslant 0\\
        &-\hat{y}_i + y_i -\xi_i - \varepsilon \leqslant 0\\
        &\alpha_i \xi_i = 0\\
        &\beta_i (\hat{y}_i - y_i - \xi_i - \varepsilon) = 0\\
        &\gamma_i (-\hat{y}_i + y_i -\xi_i - \varepsilon) = 0
    \end{cases}
\end{equation}
If $\beta_i > 0$, then $\hat{y}_i - y_i -\xi_i - \varepsilon = 0$, from this we can get
\begin{equation}
    b = y_i + \xi_i + \varepsilon - \sum_{j=1}^m(\gamma_j-\beta_j)\langle\V{x}_j, \V{x}_i\rangle
\end{equation}
Similarly, if $\gamma_i > 0$, then $-\hat{y}_i + y_i -\xi_i - \varepsilon = 0$, which implies
\begin{equation}
    b = y_i - \xi_i - \varepsilon - \sum_{i=1}^m (\gamma_j - \beta_j)\langle\V{x}_j, \V{x}_i\rangle
\end{equation}
To get the bias $b$ of the predictor, we can first calculate all $b$\,s by the above two equations, then use
their average as the bias.

% ========Neuron Network
\chapter{Neuron Networks}
\section{Neuron Network}

% a 2 layer neuron network
\begin{figure}[h]
    \begin{center}
    \begin{tikzpicture}
        % layer 0: input layer
        \begin{scope}[name prefix=layer0-]
            \node at (0, 0) {Layer 0};
            \node[circle, draw] (1) at (0, -1) {$x_1$};
            \node[circle, draw] (2) at (0, -2.5) {$x_2$};
            \node[circle, draw] (3) at (0, -4) {$x_3$};
        \end{scope}

        % layer 1
        \begin{scope}[name prefix=layer1-]
            \node at (3, 0) {Layer 1};
            \node[circle, draw] (1) at (3, -1) {$a^1_1$};
            \node[circle, draw] (2) at (3, -2) {$a^1_2$};
            \node[circle, draw] (3) at (3, -3) {$a^1_3$};
            \node[circle, draw] (4) at (3, -4) {$a^1_4$};
        \end{scope}

        % layer 2
        \begin{scope}[name prefix=layer2-]
            \node at (6, 0) {Layer 2};
            \node[circle, draw] (1) at (6, -1.5) {$a^2_1$};
            \node[circle, draw] (2) at (6, -3.5) {$a^2_2$};
        \end{scope}

        % weights
        \foreach \x in {1, 2, 3} {
            \foreach \y in {1, 2, 3, 4} {
                \foreach \z in {1, 2} {
                    \draw[gray, ->] (layer0-\x) -- (layer1-\y);
                    \draw[gray, ->] (layer1-\y) -- (layer2-\z);
                }
            }
        }
        \draw (layer0-1) edge[red, ->] node[auto] {$w^1_{12}$} (layer1-2);
        \draw (layer0-3) edge[red, ->] node[auto] {$w^1_{32}$} (layer1-2);
        \draw (layer1-1) edge[red, ->] node[auto] {$w^2_{12}$} (layer2-2);
        \draw (layer1-4) edge[red, ->] node[auto] {$w^2_{42}$} (layer2-2);
    \end{tikzpicture}
    \end{center}
    \caption{A 2 Layer Neuron Network}
\end{figure}

\subsection{Notations}
\begin{description}
    \item[$L$] output layer.
    \item[$n_l$] number of neurons in layer $l$. In particular, $n_0 = n$.
    \item[$w^l_{ij}$] weight from the $i$-th neuron in the layer $l-1$ to the $j$-th neuron in the layer $l$.
    \item[$b^l_j$] bias of the $j$-th neuron in layer $l$.
    \item[$a^l_j$] activation of the $j$-th neuron in layer $l$.
    \item[$z^l_j$] raw output of the $j$-th neuron in layer $l$.
    \item[$\sigma^l_j$] activation function of the $j$-th neuron in layer $l$.
    \item[$w^l$] weight matrix connecting layer $l-1$ to layer $l$, i.e. $(w^l_{ij})$, of dimension 
    $(n_{l-1},\ n_l)$.
    \item[$b^l$] bias vector of layer $l$, i.e. $(b^l_j)$, of dimension $(1, n_l)$.
    \item[$z^l$] raw output vector of layer $l$, of dimension $(1, n_l)$.
    \item[$a^l$] activation vector of layer $l$, of dimension $(1, n_l)$.
    \item[$\sigma^l$] activation function vector of layer $l$, of dimension $(1, n_l)$.
\end{description}

Some basic equations:

\begin{align}
    z^0_j &= a^0_j = x_j\\
    z^l_j &= \sum_k a^{l-1}_{k} w^l_{kj} + b^l_j\quad\forall~l\geq 1\\
    a^l_j &= \sigma^l_j(z^l_j)
\end{align}
The corresponding matrix forms are:

\begin{align}
    z^0 &= a^0 = \V{x} = (x_1, x_2, \ldots, x_{n})\\
    z^l &= a^{l-1} w^l + b^l\\
    a^l &= \sigma^l(z^l)
\end{align}

For a single input example \V{x}, the cost function $C$ should only directly depend on the output layer $L$,
for example $C$ is the square loss function:
\begin{equation}\label{nn_square_loss}
    C = \frac{1}{2} ||a^L - y||^2_2
\end{equation}
For a collection of examples, the cost function is the average cost on those examples:
\begin{equation}
    C = \frac{1}{m} \sum_{i=1}^m C(\V{x}_i)
\end{equation}

\subsection{Backpropagation}
Let $\delta^l_j$ be the error in the $j$-th neuron in the $l$-th layer, i.e.
\begin{equation}
    \delta^l_j = \pfrac{C}{z^l_j}
\end{equation}

For the output layer $L$, by definition we have:
\begin{align*}
    \delta^l_j &= \pfrac{C}{z^l_j}\\
               &= \sum_k \pfrac{C}{a^L_k} \pfrac{a^L_k}{z^L_j}\\
               &= \pfrac{C}{a^L_j} \pfrac{\sigma^L_j(z^L_j)}{z^L_j}\\
               &= \pfrac{C}{a^L_j} (\sigma^L_j)'(z^L_j)
\end{align*}
That is,
\begin{equation}
    \delta^L = \nabla_{a^L}C \odot (\sigma^L)'(z^L)
\end{equation}
When $C$ is the square loss~\eqref{nn_square_loss}, $\nabla_{a^L}C = a^L - y$.
\par
We can write $\delta^l$ in terms of $\delta^{l+1}$ as following:
\begin{align*}
    \delta^l_j &= \pfrac{C}{z^l_j}\\
               &= \sum_k \pfrac{C}{z^{l+1}_k} \pfrac{z^{l+1}_k}{z^l_j}\\
               &= \sum_k \delta^{l+1}_k \sum_r \pfrac{a^l_r}{z^l_j} w^{l+1}_{rk}\\
               &= \sum_k \delta^{l+1}_k (\sigma^l_j)'(z^l_j) w^{l+1}_{jk}\\
               &= {\left(\delta^{l+1} \T{(w^{l+1})}\right)}_{j} (\sigma^l_j)'(z^l_j)
\end{align*}
Its corresponding matrix form is:
\begin{equation}
    \delta^l = \left(\delta^{l+1} \T{(w^{l+1})}\right) \odot (\sigma^l)'(z^l)
\end{equation}

Now, let's compute \pfrac{C}{b^l_j}:
\begin{align*}
    \pfrac{C}{b^l_j} &= \sum_k \pfrac{C}{z^l_k} \pfrac{z^l_k}{b^l_j}\\
                     &= \sum_k \delta^l_k \pfrac{b^l_k}{b^l_j}\\
                     &= \delta^l_j
\end{align*}
In shorthand, it can be rewritten as:
\begin{equation}
    \pfrac{C}{b} = \delta
\end{equation}

Similarly, we can compute \pfrac{C}{w^l_{ij}}:
\begin{align*}
    \pfrac{C}{w^l_{ij}} &= \sum_k \pfrac{C}{z^l_k} \pfrac{z^l_k}{w^l_{ij}}\\
                        &= \sum_k \delta^l_k \sum_r \pfrac{(a^{l-1}_r w^l_{rk} + b^l_k)}{w^l_{ij}}\\
                        &= \sum_k \delta^l_k a^{l-1}_i \delta_{kj}\\
                        &= a^{l-1}_i \delta^l_j
\end{align*}
In shorthand, it can be rewritten as:
\begin{equation}
    \pfrac{C}{w} = a_{\text{in}}\delta_{\text{out}}
\end{equation}

For simplicity, let's assume that all the activation functions are the same, i.e. $\sigma^l_i = \sigma$, then 
we can write the pseudocode of backpropagation algorithm easily as the following:

\begin{algorithm}
    \caption{Backpropagation}\label{backpropagation}
    \begin{algorithmic}[1]
        \Require $\V{x} = (x_1, x_2, \ldots, x_n)$
        \For{$l = 1$ \algorithmicto $L$}
            \State Compute $z^l = a^{l-1} w^l + b^l$ and $a^l = \sigma(z^l)$.
        \EndFor
        \State Compute $\delta^l = \nabla_{a^L}C \odot \sigma'(z^L)$\Comment{$\nabla_{a^L}C = a^L - y$ if $C$ 
        is square loss.}
        \For{$l=L-1$ \algorithmicto $1$}
            \State Compute $\delta^l = \left(\delta^{l+1} \T{(w^{l+1})}\right) \odot \sigma'(z^l)$
        \EndFor
        \Ensure $\pfrac{C}{w^l_{ij}} = a^{l-1}_i \delta^l_j$ and $\pfrac{C}{b^l_j} = \delta^l_j$.
    \end{algorithmic}
\end{algorithm}

This is the algorithm for a single example, we are now ready to do the vectorization. Let $\V{x}^i$ and $y^i$
be the $i$-th example and its output respectively, let
\begin{align*}
    X &= {(\V{x}^1; \ldots; \V{x}^m)}_{m \times n}\\
    Y &= {(y^1; \ldots; y^m)}_{m \times n_L}
\end{align*}
the input matrix. 
Let $z^{i, l}, a^{i, l}, \delta^{i, l}$ be the raw output, output, error vectors w.r.t.\ the
$i$-th example respectively. Let 
\begin{align*}
    Z^l &= {(z^{1, l}; \ldots; z^{m, l})}_{m \times n_l}\\
    A^l &= {(a^{1, l}; \ldots; a^{m, l})}_{m \times n_l}\\
    \Delta^l &= {(\delta^{1, l}; \ldots; \delta^{m, l})}_{m \times n_l}
\end{align*}
then we have:
\begin{align}
    Z^l &= A^{l-1} w^l + b^l\\
    A^l &= \sigma(Z^l)\\
    \Delta^L &= \nabla_{A^L}C \odot \sigma'(Z^L)\\
    \Delta^l &= \left(\Delta^{l+1} \T{(w^{l+1})}\right) \odot \sigma'(Z^l)
\end{align}
If $C = \frac{1}{m}\sum_{i=1}^m C(\V{x}_i)$, then
$$\pfrac{C}{b^l_j} = \operatorname{reduce\_mean}(\operatorname{col}_j(\Delta^l))$$
and
$$\pfrac{C}{w^l_{ij}} = \operatorname{reduce\_mean}(\operatorname{col}_i(A^{l-1}) \odot \operatorname{col}_j
(\Delta^l))$$

% ========Bayesian Classifier=========
\chapter{Bayesian}
\include{part_1/bayes/bayesian_classifier}

% ========Clustering=================
\chapter{Clustering}
\include{part_1/clustering/clustering}

% ========Expectation Maximalization====
\chapter{Expectation Maximalization}
\include{part_1/em/EM}

% ========Ensemble Learning============
\chapter{Ensemble Learning}
\section{AdaBoost}
AdaBoost combines a family of base (weak) learners from an algorithm linearly to get a stronger (more
powerful) one. It takes advantage
of a former learner and use its output to force the latter learner to focus more on those data on which the
former learner performs poorly. In this way, it hopes to make the aggregate learner performs well on all
training data. Thus AdaBoost focus more on reducing the bias. 
% AdaBoost algorithm
\begin{algorithm}
    \caption{AdaBoost}\label{AdaBoost}
    \begin{algorithmic}[1]
        \Require training set $D = \{(\V{x}_1, y_1), \ldots, (\V{x}_m, y_m)\}$; base learner $\mathcal{L}$;
        training round $T$.
        \State $\mathcal{D}_1(\V{x}) = \frac{1}{m}$.\Comment{$\mathcal{D}$ is a distribution over the input 
        examples.}
        \For{$t = 1$ \algorithmicto $T$}
            \State $h_t = \mathcal{L}(D, \mathcal{D}_t)$;
            \State $\varepsilon_t = \p_{\V{x}\sim\mathcal{D}_t}(h_t(\V{x})\neq f(\V{x}))$;\Comment{
            $\varepsilon_t$ is the error rate of $h_t$ w.r.t $\mathcal{D}_t$.}
            \State $\alpha_t = \frac{1}{2}\ln{\frac{1 - \varepsilon_t}{\varepsilon_t}}$;\label{AdaBoost_alphat}
            \State $\mathcal{D}_{t + 1}(\V{x}) = \frac{\mathcal{D}_t(\V{x})\exp(-\alpha_t f(\V{x})h_t(\V{x}))}
            {Z_t}$.\Comment{$Z_t$ is the normalization factor.}\label{AdaBoost_distritution}
        \EndFor
        \Ensure $H(\V{x}) = \sign(\displaystyle\sum_{t = 1}^T \alpha_t h_t(\V{x}))$
    \end{algorithmic}
\end{algorithm}

%TODO: equivalent to 0/1 loss function.
Consider the exponential loss function
\begin{equation}
    L_{\exp}(H | \mathcal{D}) = \E_{\V{x}\sim\mathcal{D}}[e^{-f(\V{x})H(\V{x})}]
\end{equation}
We want to find a linear combination of base learners 
$$ H(\V{x}) = \sum_{t=1}^T \alpha_t h_t(\V{x})$$
that minimize it. Although it is difficult to find such $H$ as a whole, we can approach this problem step by
step. That is, we first find some $h_1$ and $\alpha_1$ \st $H_1 = \alpha_1 h_1$ minimize the exponential loss,
then we find some $h_2$ and $\alpha_2$ \st $H_2 = H_1 + \alpha_2 h_2$ minimize the exponential loss too. And
we carray on this process until $t = T$. Assume we have $H_{t-1}$ and want to find the next learner and its
weight, that is, we want to find $\alpha_t$ and $h_t$ \st $H_t = H_{t-1} + \alpha_t h_t$ minimize the 
exponential loss. Since
\begin{align*}
    L_{\exp}(H_t | \mathcal{D}) &= L_{\exp}(H_{t-1} + \alpha_t h_t | \mathcal{D})\\
    &= \E_{\V{x}\sim \mathcal{D}}[e^{-f(\V{x})H_{t-1}(\V{x})}\cdot e^{-f(\V{x})\alpha_t h_t(\V{x})}]\\
    &= C_{t-1} \E_{\V{x}\sim \mathcal{D}}[\frac{e^{-f(\V{x})H_{t-1}(\V{x})}}{C_{t-1}}
    e^{-f(\V{x})\alpha_t h_t(\V{x})}]\\
    &= C_{t-1} \E_{\V{x}\sim \mathcal{D}_t}[e^{-f(\V{x})\alpha_t h_t(\V{x})}]
\end{align*}
where $C_{t-1} = \E_{\V{x}\sim\mathcal{D}}[e^{-f(\V{x})H_{t-1}(\V{x})}]$ is a constant and 
$\mathcal{D}_t = \frac{e^{-f(\V{x})H_{t-1}(\V{x})}}{C_{t-1}} \mathcal{D}$ is a distribution, we have
\begin{align}
    L_{\exp}(H_t | \mathcal{D}) &= C_{t-1} \left\{e^{-\alpha_t} \p_{\V{x}\sim\mathcal{D}_t}(f(\V{x})=h_t(\V{x})) 
    + e^{\alpha_t}\p_{\V{x}\sim\mathcal{D}_t}(f(\V{x})\neq h_t(\V{x}))\right\}\nonumber\\
    &= C_{t-1} \{e^{-\alpha_t}(1 - \varepsilon_t) + e^{\alpha_t}\varepsilon_t\}\label{Ada_exp_loss_expansion}
\end{align}
where $\varepsilon_t$ is the error rate of $h_t$ w.r.t.\ the distribution $\mathcal{D}_t$. Hence 
$$\pfrac{ }{\alpha_t}L_{\exp}(H_t | \mathcal{D}) = 0$$
implies 
\begin{equation}\label{AdaBoost_weight}
    \alpha_t = \frac{1}{2}\ln(\frac{1-\varepsilon_t}{\varepsilon_t})
\end{equation}
which justifies the choice of $\alpha_t$ in line~\algref{AdaBoost}{AdaBoost_alphat}. Moreover, 
$$\mathcal{D}_t = \frac{e^{-f(\V{x})H_{t-1}(\V{x})}}{C_{t-1}} \mathcal{D}$$
implies
\begin{align}
\mathcal{D}_{t+1}(\V{x}) &= \frac{e^{-f(\V{x})\alpha_t h_t(\V{x})}C_{t-1}\mathcal{D}_t(\V{x})}{C_t}\nonumber\\
                         &= \frac{e^{-f(\V{x})\alpha_t h_t(\V{x})}\mathcal{D}_t(\V{x})}{Z_t}
\end{align}
which justifies the choice of $\mathcal{D}_t$ in line~\algref{AdaBoost}{AdaBoost_distritution}. If we plug
equation~\eqref{AdaBoost_weight} into~\eqref{Ada_exp_loss_expansion}, we get
$$L_{\exp}(H_t | \mathcal{D}) = 2C_{t-1}\cdot\sqrt{\varepsilon_t (1-\varepsilon_t)}$$
Since the performance of the base learner is slightly above the average, that is, 
$\varepsilon_t < \frac{1}{2}$, the minimum of the exponential loss is achieved at the $h_t$ which has 
the minimal error rate w.r.t.\ $\mathcal{D}_t$. That is, $h_t$ is the output of the algorithm w.r.t.\ the 
distribution $\mathcal{D}_t$ over the training data.


\section{Bagging}

% Bagging algorithm
\begin{algorithm}
    \caption{Bagging}\label{Bagging}
    \begin{algorithmic}[1]
        \Require training set $D = \{(\V{x}_1, y_1), \dotsc, (\V{x}_m, y_m)\}$; base learner $\mathcal{L}$;
        training round $T$.
        \For{$t = 1$ \algorithmicto $T$}
            \State $h_t = \mathcal{L}(D, \mathcal{D}_{bs})$\Comment{bs stands for bootstrap sampling.} 
        \EndFor
        \Ensure $H(\V{x}) = \displaystyle\argmax_{y\in \mathcal{Y}}\sum_{t=1}^T 1(h_t(\V{x}) = y)$\Comment{$H$
        is the plurality vote of $h_t$.}
    \end{algorithmic}
\end{algorithm}

\section{Random Forest}

% 

% ========Dimension Reduction=========
\chapter{Dimension Reduction}
\section{Dimension Reduction}

\subsection{Low-dimensional embedding}
Let the feature space $\mathcal{X} = \mathbb{R}^d$. Our goal is to find an embedding of all samples into a
low dimension space $\mathbb{R}^{d'}$, here $d' \leq d$. That is, we want to find a map $e: \mathbb{R}^d
\longrightarrow \mathbb{R}^{d'}$ \st for any samples $\V{x}_i, \V{x}_j$, we have 
$$\dist(e(\V{x}_i), e(\V{x}_j)) = \dist(\V{x}_i, \V{x}_j)$$
Thus ${\left\{e(\V{x}_i)\right\}}_i$ is a low-dimensional embedding of the original samples. 

Let $\V{D} = {\left(\dist(\V{x}_i, \V{x}_j)\right)}_{m \times m}$ be the distance matrix of the samples 
${\left\{\V{x}_i\right\}}_i$, we want to find the embedding matrix $\V{Z} \in \mathbb{R}^{m \times d'}$, \st 
$||\V{z}_i - \V{z}_j|| = \V{D}_{ij}$, where $\V{Z} = (\V{z}_1; \dotsc; \V{z}_m)$ and 
$\V{D}_{ij} = \dist(\V{x}_i, \V{x}_j)$. Hence we have 
$$ \V{D}_{ij}^2 = ||\V{z}_i||^2 + ||\V{z}_j||^2 - 2 \langle\V{z}_i, \V{z}_j\rangle$$
Let $\V{B} = {\left(b_{ij}\right)}_{m \times m}= \V{Z}\T{\V{Z}}$ where $b_{ij} = \langle\V{z}_i, \V{z}_j\rangle$, 
we have
\begin{equation}\label{DR_Dij}
    \V{D}_{ij}^2 = b_{ii} + b_{jj} - 2 b_{ij}
\end{equation}
Let $\V{c} = \frac{1}{m}\sum_i \V{z}_i$, then ${\left\{\V{z}_i - \V{c}\right\}}_i$ is an embedding with the
property that $\sum_i(\V{z}_i - \V{c}) = 0$. Hence we can require that $\sum_i \V{z}_i = 0$ in the above
discussion. Then from equation~\eqref{DR_Dij}, we know that:
\begin{align}
    \sum_i \V{D}_{ij}^2 &= \tr(\V{B}) + m b_{jj}\\
    \sum_j \V{D}_{ij}^2 &= \tr(\V{B}) + m b_{ii}\\
    \sum_{i,j} \V{D}_{ij}^2 &= 2m \tr(\V{B})
\end{align}
From the above equations, we have
\begin{equation}\label{DR_bij}
    b_{ij} = -\frac{1}{2}\left(\V{D}_{ij}^2 - \frac{1}{m}\sum_i \V{D}_{ij}^2 - \frac{1}{m} \sum_j \V{D}_{ij}^2
    + \frac{1}{m^2}\sum_{i,j}\V{D}_{ij}^2\right)
\end{equation}
That is, $\V{B}$ is totally determined by $\V{D}$. Let $\V{B} = \V{V}\V{\Lambda}\T{\V{V}}$ be the eigenvalue
decomposition of $\V{B}$. Since $\V{B}$ is semi-positive, $\V{\Lambda}$ is a diagonal matrix with non-negative
diagonals. Let $\V{\Lambda}_*$ be the diagonal matrix obtained by removing the zero eigenvalues from 
$\V{\Lambda}$ and $\V{V}_*$ the matrix by removing the corresponding columns of $\V{V}$. Then it's easy to 
conclude that 
$$\V{Z} = \V{V}_*\V{\Lambda}_*^{\nicefrac{1}{2}}$$
is what we want. Hence the dimension $d'$ is totally determined by the distance matrix $\V{D}$. 

In practise, we often fix some $d' \ll d$ at first, then obtain $\V{\Lambda}_*$ by keeping the $d'$ largest 
eigenvalues and remove the rest, and $\V{V}_*$ the corresponding matrix. By this way, we may lose some 
precision in keeping the pairwise distance, but we can greatly reduce the dimension.

\begin{algorithm}
    \caption{Multiple Dimensional Scaling}
    \begin{algorithmic}[1]
        \Require the distance matrix $\V{D}_{m \times m}$; dimension $d'$.
        \State Compute the matrix $\V{B}$ according to equation~\eqref{DR_bij}.
        \State Eigenvalue decomposition: $\V{B} = \V{V} \V{\Lambda}\T{\V{V}}$.
        \State Let $\V{\Lambda}_*$ be the diagonal matrix with the $d'$ largest eigenvalues and $\V{V}_*$ the
        matrix with the corresponding eigenvectors.
        \Ensure Low-dimensional embedding: $\V{Z} = {\left(\V{V}_* \V{\Lambda}_*^{\nicefrac{1}{2}}\right)}_{m 
        \times d'}$
    \end{algorithmic}
\end{algorithm}

\subsection{Linear Dimension Reduction}
The simplest way to reduce dimension is by dropping some coordinates, that is, via projection. This keeps the
linear structure of the samples. A more generalized way is via linear transformation: $\V{X} \V{W} = \V{Z}$,
where $\V{X} = {\left(\V{x}_1; \dotsc; \V{x}_m\right)}_{m \times d}$ is the feature matrix of the samples, $\V{W} \in 
\mathbb{R}^{d \times d'}$ the transformation matrix, and $\V{Z}_{m \times d'}$ the representation matrix
in low dimensional space. If we write $\V{Z} = {\left(\V{z}_1; \dotsc; \V{z}_{m}\right)}$, then $\V{z}_i =
\V{x}_i \V{W}$.




% ========Computational Learning Theory
\chapter{Computational Learning Theory}
\section{Computational Learning Theory}
In this section, we mainly consider supervised learning.
Let $\mathcal{X}$ be the instance space, $\mathcal{Y}$ the label set, $D = \{(\V{x}_1, y_1), \ldots, 
(\V{x}_m, y_m)\}$ the training set. Assume $\mathcal{D}$ is the distribution on $\mathcal{X}$, and all 
instances of $D$ are sampled i.i.d.\ according to $\mathcal{D}$. Let \hypo{f} be the underlying labeling 
function and \hypo{h} any prediction function, then the \textbf{true (generalization) loss (error)} is defined
as:
$$L_{\mathcal{D}, f}(h) := \p_{\V{x}\sim\mathcal{X}}(h(\V{x})\neq f(\V{x})) := \mathcal{D}\left(\{\V{x}: h(\V{x})
\neq f(\V{x})\}\right)$$
the \textbf{empirical risk (error, loss)} is defined as:
$$L_S(h) = \frac{1}{m}\sum_{i=1}^m \indi(h(\V{x})\neq f(\V{x}))$$

\subsection{Probably Approximately Correct Learning}

% concept class.
\begin{df}[Concept class]
    A (target) concept is just a true labeling function \hypo{c}, that is, for any instace $(\V{x}, y)$ 
    (assuming sampling process is noise free) we have $c(\V{x}) = y$. The collection $\mathcal{C}$ of all 
    target concepts is called the concept class.
\end{df}

% hypothesis space.
\begin{df}[Hypothesis space]
    The collection of all labeling functions \hypo{f} a learner $\mathcal{L}$ can return is called the 
    hypothesis space (w.r.t.\ $\mathcal{L}$). We denote it as $\mathcal{H}$.
\end{df}

% inductive bias.
\begin{re}[Inductive bias]
    By restricting our learner to the hypothesis space instead of arbitrary predictors, we bias it toward a 
    particular set of predictors. Such restrictions are called \textbf{inductive bias}.
\end{re}

% realizability assumption.
\begin{df}[Realizability Assumption]
    The realizable assumption asserts that there is a $h^* \in \mathcal{H}$ s.t.\ 
    $L_{\mathcal{D}, f}(h^*) = 0$.
\end{df}

\begin{re}
    The realizable assumption implies that with probability 1 over i.i.d.\ samples $D$, we have $L_D(h^*) = 0$.
    That is, $\mathcal{D}^m(\{D: L_D(h^*) = 0\}) = 1$.
\end{re}

% PAC learnability.
\begin{df}[PAC learnability]\label{PAC_learnability}
    The concept class $\mathcal{C}$ is PAC learnable w.r.t.\ a hypothesis space $\mathcal{H}$ if there exist
    \begin{enumerate}
        \item a function $m: {(0, 1)}^2 \longrightarrow \mathbb{N}$;
        \item a learner $\mathcal{L}$.
    \end{enumerate}
    s.t.\ for any $\varepsilon, \delta \in (0, 1)$, for any distribution $\mathcal{D}$ over $\mathcal{X}$, and
    for any concept \hypo{c}, if the realizable assumption holds w.r.t.\ $\mathcal{H}, \mathcal{D}, c$, then
    when applying the learner $\mathcal{L}$ to $m \geq m(\varepsilon, \delta)$ i.i.d.\ samples generated by
    $\mathcal{D}$ and labeled by $c$, the learner returns a hypothesis $h$ s.t.\ with probability at least 
    $1 - \delta$ (over the choice of the samples), we have $L_{\mathcal{D},c}(h) \leq \varepsilon$. That is,
    $$\mathcal{D}^m(\{D: L_{\mathcal{D},c}(h) \leq \varepsilon\}) \geq 1 - \delta$$
\end{df}

% sample complexity.
\begin{df}[Sample complexity]
    The sample complexity of a learner is the minimal number of examples needed for the learner to produce a 
    PAC solution on any i.i.d.\ data sets with that many samples. That is, it is the minimum of all 
    $m(\varepsilon, \delta)$ where $m$ satisfies the requirements in definition~\ref{PAC_learnability}.
\end{df}

% agnostic PAC learnability.

% ========Reinforcement Learning
\chapter{Reinforcement Learning}
\section{Reinforcement Learning}

Let $X$ be the state space, $A$ the action space. Reinforcement learning can be described as a
\textbf{Markov Decision Process}: system changes from a state to another under the actions from $A$ with a 
rewarding function grading the change. That is, a reinforcement learning is a quadruple $E = \langle X, A, P,
R\rangle$, where $P: X \times A \times X \longrightarrow \mathbb{R}$ describes the probability of a state
changed into another via an action from $A$ and $R: X \times A \times X \longrightarrow \mathbb{R}$ describes
the reward of that change. Sometimes the rewarding function only depends on states, that is $R: X \times X
\longrightarrow \mathbb{R}$. Note that given a state $x$ and an action $a$ that can act on it, the resulting 
state $x'$ may not be unique, but the identity $\sum_{x'\in X}P_{x\rightarrow x'}^a = 1$ always holds true.
\par
The goal of reinforcement leanring is to find a \hl{policy $\pi$} dictating which action to take given a state
$x$. $\pi$ can be described in a determinate form $\pi: X \longrightarrow A$, which dictates that action
$\pi(x)$ must be performed on state $x$. It can also be described in a probability form: $\pi: X \times A
\longrightarrow \mathbb{R}$ where $\pi(x, a)$ is the probability of performing action $a$ on $x$. Obviously,
the determinate form is a special case of the probability form. The performance measurement of the policy is 
the accumulated reward gained from performing this policy for a reasonably long time. The \textbf{$T$ steps
accumulated reward} $\E[\frac{1}{T}\sum_{t=1}^{T}r_t]$ and \textbf{$\gamma$ discount accumulated reward}
$\E[\sum_{t=0}^{\infty}\gamma^t r_{t+1}]$ are two commonly used accumulated rewards, where $r_t$ is the reward
of step $t$.

\subsection{K-armed Bandit}
The K-armed bandit is a model of one step reinforcement learning. Every time you pull one arm of the machine,
it gives you some rewards with certain probability. The goal is to acquire as many rewards as possible.

\subsubsection{Exploration and Exploitation}
The \textbf{exploration-only} method tries to figure out the expectation reward of each arm. It divides the 
exploration chance evenly on each arm, and use the average reward of each arm as the expectation reward. This
method may do a good job in estimating each arm's reward, but it may miss the opportunity to get the most 
rewards. The \textbf{exploitation-only} method, on the other hand, only pulls the arm that gives the most 
average rewards so far. Since it dosen't care about the expectation reward of each arm, it may also miss the
opportunity to get the most rewards. 

\subsubsection{$\varepsilon$-greedy}
The $\varepsilon$-greedy is a compromise between exploration-only and exploitation-only method. At
each try, it performs exploration-only method with a probability $\varepsilon$ and exploitation-only method
the other $1 - \varepsilon$.\par
Let $Q_n(k)$ denote the average reward of arm $k$ with $n$ tries and $v_n^k$ its n-th reward, then it is 
clearly that
$$Q_n(k) = \frac{1}{n} ((n-1)Q_{n-1}(k) + v_n^k)$$
This is the updating rules for the average reward.

% epsilon-greedy algorithm
\begin{algorithm}
    \caption{$\varepsilon$-greedy}\label{epsilon_greedy_for_K_arm_bandit}
    \begin{algorithmic}[1]
        \Require number of arms $K$; rewarding function $R$; number of tries $T$; exploration threshold 
        $\varepsilon$.
        \State $r = 0$\Comment{the accumulated reward.}
        \State $Q(i) = 0, \quad count(i) = 0 \quad \forall~i=1,\ldots,K$\Comment{Initialization.}
        \For{$t = 1$ \algorithmicto $T$}
            \If{$rand() < \varepsilon$}
                \State $k = rand(\{1, \ldots, K\})$\Comment{choose with equal probability}
            \Else
                \State $k = \argmax_i Q(i)$
            \EndIf
            \State $v = R(k)$
            \State $r \gets r + v$
            \State $Q(k) \gets \frac{count(k) \cdot Q(k) + v}{count(k) + 1}$
            \State $count(k) \gets count(k) + 1$
        \EndFor
        \Ensure the accumulated reward $r$.
    \end{algorithmic}
\end{algorithm}

% TODO: analysis of the epsilon-greedy.

\subsubsection{Softmax}
Softmax algorithm is another way of compromising between the exploration-only and exploitation-only methods. 
Unlike the $\varepsilon$-greedy algorithm which uses a threshold to determine how to choose an arm, softmax 
attach each arm with a probability to be chosen base on its current average reward. The probability 
distribution among the arms is a Boltzmann distribution, namely:
\begin{equation}\label{K_arm_bandit_softmax}
P(k) = \frac{e^{\frac{Q(k)}{\tau}}}{\sum_{i=1}^K e^{\frac{Q(i)}{\tau}}}\quad\forall~k = 1,\ldots,K
\end{equation}
where the parameter $\tau > 0$. Apparently, if $\tau$ is close to $0$, the softmax favours the arm with
the highest average reward, which is the case of exploitation-only method; if $\tau$ is very large (close to 
$+\infty$), the softmax degenerates to uniform distribution, which is the case of exploration-only method.

% softmax algorithm for K-arm bandit.
\begin{algorithm}
    \caption{Softmax}\label{Softmax_for_K_arm_bandit}
    \begin{algorithmic}[1]
        \Require number of arms $K$; rewarding function $R$; number of tries $T$; parameter $\tau$.
        \State $r = 0$
        \State $Q(i) = 0, \quad count(i) = 0\quad\forall~i=1,\ldots, K$
        \For{$t = 1$ \algorithmicto $T$}
            \State choose $k$ according to the distribution given by equation~\eqref{K_arm_bandit_softmax}.
            \State $v = R(k)$
            \State $r \gets r + v$
            \State $Q(k) \gets \frac{count(k)\cdot Q(k)}{count(k) + 1}$
            \State $count(k) \gets count(k) + 1$
        \EndFor
        \Ensure the accumulated reward $r$.
    \end{algorithmic}
\end{algorithm}

% TODO: analysis of the softmax algorithm.

\newpage
\subsection{Model-based Learning}
In model-based learning, the quadruple $E = \langle X, A, P, R\rangle$ are known to us. That is, we know all
the possible states, all the possible actions that may act on them, the transition function which describes
the probability of one state changing into another under some action, the rewarding function which grades the
change. In the following discussion, we assume that \magenta{the state space $X$ and the action space $A$ are
finite}.

\subsubsection{Policy Measurement}
Let $V^\pi(x)$ be the expected reward gained from applying policy $\pi$ on the starting state $x$, 
$Q^\pi(x, a)$ the expected reward gained from first applying action $a$ then policy $\pi$ on the starting
state $x$. $V$ and $Q$ are called \textbf{state value function} and \textbf{state-action value function}
respectively. \par
By definition, we can write the state value function as:
\begin{align}
    V^\pi_T(x) &= \E_\pi\left[\frac{1}{T}\sum_{t=1}^{T}r_t | x_0 = x\right], &\text{$T$ steps}
    \label{state_value_T}\\
    V^\pi_\gamma(x) &= \E_\pi\left[\sum_{t=0}^{\infty} \gamma^t r_{t+1} | x_0 = x\right], &\text{$\gamma$ 
    discount}\label{state_value_gamma}
\end{align}
Similarly, we can write the state-action value function as:
\begin{align}
    Q^\pi_T(x, a) &= \E_\pi\left[\frac{1}{T}\sum_{t=1}^T r_t | x_0 = x, a_0 = a\right], &\text{$T$ steps}
    \label{state-action_value_T}\\
    Q^\pi_\gamma(x, a) &= \E_\pi\left[\sum_{t=0}^\infty \gamma^t r_{t+1} | x_0 = x, a_0 = a\right] &\text{
    $\gamma$ discount}\label{state-action_value_gamma}
\end{align}
Since the next state of the system only depends on the current state, those value functions can be written in
a recursive form. For example, the $T$ steps accumulated state value function~\eqref{state_value_T} can be 
written as:
\begin{align}
    V^\pi_T(x) &= \E_\pi \left[\frac{1}{T}\sum_{t=1}^T r_t | x_0 = x\right]\nonumber\\
               &= \E_\pi \left[\frac{1}{T} r_1 + \frac{T-1}{T}\frac{1}{T-1}\sum_{t=2}^T r_t | x_0 = x\right]
               \nonumber\\
               &= \sum_{a\in A}\pi(x, a)\sum_{x'\in X}P_{x\rightarrow x'}^a\left(\frac{1}{T}
               R_{x\rightarrow x'}^a + \frac{T-1}{T}\E_\pi\left[\frac{1}{T-1}\sum_{t=1}^{T-1}r_t | x_0=x'
               \right]\right)\nonumber\\
               &= \sum_{a\in A}\pi(x, a)\sum_{x'\in X}P_{x\rightarrow x'}^a\left(\frac{1}{T}
               R_{x\rightarrow x'}^a + \frac{T-1}{T} V^\pi_{T-1}(x')\right)
\end{align}
Similarly, the $\gamma$ discount accumulated state value function~\eqref{state_value_gamma} can be written as:
\begin{align}
    V^\pi_\gamma(x) &= \E_\pi\left[\sum_{t=0}^\infty \gamma^t r_{t+1} | x_0 = x\right]\nonumber\\
                    &= \E_\pi\left[r_0 + \gamma \sum_{t=1}^\infty \gamma^{t-1}r_{t+1} | x_0=x\right]\nonumber\\
                    &= \sum_{a\in A}\pi(x, a)\sum_{x'\in X}P_{x\rightarrow x'}^a \left(R_{x\rightarrow x'}^a + 
                    \gamma \E_\pi\left[\sum_{t=0}^\infty \gamma^t r_{t+1} | x_0 = x'\right]\right)\nonumber\\
                    &= \sum_{a\in A}\pi(x, a)\sum_{x'\in X}P_{x\rightarrow x'}^a \left(R_{x\rightarrow x'}^a +
                    \gamma V^\pi_\gamma(x')\right)
\end{align}

% algorithm for calculating T steps accumulated state value function.
\begin{algorithm}
    \caption{$T$ steps accumulated state value function}
    \begin{algorithmic}[1]
        \Require $E = \langle X, A, P, R\rangle$; policy $\pi$; steps $T$.
        \State $V(x) = 0\quad\forall~x\in X$
        \For{$t=1$ \algorithmicto $T$}
            \State $\displaystyle V'(x) = \sum_{a\in A}\pi(x,a)\sum_{x'\in X}P_{x\rightarrow x'}^a 
            \left(\frac{1}{t} R_{x\rightarrow x'}^a + \frac{t-1}{t}V(x')\right)\quad\forall~x\in X$
            \If{$t = T+1$}
                \State \algorithmicbreak
            \Else
                \State $V \gets V'$
            \EndIf
        \EndFor
        \Ensure state value function $V$.
    \end{algorithmic}
\end{algorithm}

% algorithm for gamma discount accumulated state value function.
\begin{algorithm}
    \caption{$\gamma$ discount accumulated state value function}
    \begin{algorithmic}[1]
        \Require $E = \langle X, A, P, R\rangle$; policy $\pi$; parameter $\gamma$; threshold $\theta$.
        \State $V(x) = 0\quad\forall~x\in X$
        \Loop
            \State $\displaystyle V'(x) = \sum_{a\in A}\pi(x,a)\sum_{x'\in X}P_{x\rightarrow x'}^a 
            \left(R_{x\rightarrow x'}^a + \gamma V(x')\right)\quad\forall~x\in X$
            \If{$\displaystyle\max_{x\in X}|V(x) - V'(x)| < \theta$}
                \State \algorithmicbreak
            \Else
                \State $V \gets V'$
            \EndIf
        \EndLoop
        \Ensure state value function $V$.
    \end{algorithmic}
\end{algorithm}

% TODO: formulas for state-action value functions.

% TODO: algorithms for state-action value functions.

\subsubsection{Policy Improvement}

\subsection{Model-free Learning}

% =============================Part Two: Papers====================
\part{Selected Papers}

\chapter{Neuron Networks}
\section{Multilayer Feedforward Networks are Universal Approximators}
This paper\cite{hornik_1989} proves that standard multilayer feedforward networks with as few as one hidden 
layer using arbitrary squashing functions are capable of approximating any Borel measureble function, provided
sufficiently many hidden units are available.

\subsection{Terminology and Notation}
Let $\boldsymbol{A}^r = \left\{A: \mathbb{R}^r \longrightarrow \mathbb{R} | A(\V{x}) = \langle \V{w}, \V{x}
\rangle + b\right\}$ be the collection of all affine functions, $\boldsymbol{M}^r$ the collection of all 
Borel measureble functions, $\boldsymbol{C}^r$ the collection of all continuous functions and 
$\boldsymbol{B}^r$ the collection of all Borel sets, on $\mathbb{R}^r$.

\begin{df}[Squashing function]
    $\phi: \mathbb{R}\longrightarrow [0, 1]$ is a squashing function if and only if it is 
    non-decreasing, satifies $\displaystyle\lim_{\lambda \to \infty}\phi(\lambda) = 1$ and 
    $\displaystyle\lim_{\lambda \to -\infty}\phi(\lambda) = 0$.
\end{df}

\begin{df}
    Let $G: \mathbb{R} \longrightarrow \mathbb{R}$ be a Borel measureble function. Then 
    $\boldsymbol{\Sigma}^r(G)$ is defined as:
    $$\boldsymbol{\Sigma}^r(G) = \bigg\{f: \mathbb{R}^r \longrightarrow \mathbb{R} \Big| f(\V{x}) = 
    \sum_{j = 1}^q \beta_j G(A_j(\V{x})),\; q\in\mathbb{N}, \beta_j\in\mathbb{R}, A_j\in\boldsymbol{A}^r
    \bigg\}$$
    Obviously, functions in $\boldsymbol{\Sigma}^r(G)$ are the outputs of neuron networks which have a single
    hidden layer with activation $G$ (no activation in the output layer).
\end{df}

\begin{df}
    Let $G: \mathbb{R} \longrightarrow \mathbb{R}$ be a Borel measureble function. Then 
    $\boldsymbol{\Sigma\Pi}^r(G)$ is defined as:
    $$\boldsymbol{\Sigma\Pi}^r(G) = \bigg\{f: \mathbb{R}^r \longrightarrow \mathbb{R} \Big| f(\V{x}) = 
    \sum_{j = 1}^q \beta_j\prod_i^{l_j} G(A_i(\V{x}))\bigg\}$$
    Obviously, $\boldsymbol{\Sigma\Pi}^r(G)$ is the algebra generated by $\boldsymbol{\Sigma}^r(G)$.
\end{df}

\begin{df}
    Let $(X, \rho)$ be a metric space, $S \subseteq T \subseteq X$. Then $S$ is said to be $\rho$-dense in $T$ 
    if $\forany \varepsilon > 0, \forany t \in T, \thereis s \in S, \st \rho(s, t) < \varepsilon$.
\end{df}

\begin{df}
    $S \subseteq \boldsymbol{C}^r$ is \textbf{uniformly dense on compacta} in $\boldsymbol{C}^r$ if $\forany K
    \subseteq \mathbb{R}^r$ which is compact, $S$ is $\rho_K$-dense in $\boldsymbol{C}^r$. A sequence 
    $\{f_n\}$ converges to $f$ uniformly on compacta if $\forany K \subseteq \mathbb{R}^r$ compact, 
    $\displaystyle\lim_{n\to\infty}\rho_K(f_n, f) = 0$. Here the metric 
    $\rho_K$ is defined as:
    $$\rho_K(f, g) = \sup_{\V{x}\in K}|f(\V{x}) - g(\V{x})|\quad\forany f, g\in \boldsymbol{C}^r$$
\end{df}
\section{Approximation by Superpositions of a Sigmoidal Function}
blalba\cite{cybenko_1989}

%============================================================
\backmatter
% ========================Bibliography=======================
\addcontentsline{toc}{part}{Bibliography}
\bibliographystyle{amsplain}
\bibliography{notes}


\end{document}
\section{Support Vector Machine}
Let \dataset\ be the training set where $y_i \in\{-1, +1\}$. The SVM is an algorithm that tries to find a 
hyperplane $y = \T{\V{w}}\V{x} + b$ which separate the positive examples from the negative ones. The 
corresponding predictor is $f(\V{x}) = \sign(\T{\V{w}}\V{x} + b)$.

\subsection{Hard SVM}
First we introduce the concept of the margin of a hyperplane:
% margin of a hyperplane w.r.t. a training set.
\begin{df}[Margin]
    The margin of a hyperplane w.r.t.\ a training set is the minimal Euclidean distance between the point in the 
    training set and the hyperplane, that is:
    \begin{equation*}
    \min_i \frac{|\T{\V{w}}\V{x}_i + b|}{||\V{w}||}
    \end{equation*}
\end{df}

% analysis of the hard SVM algorithm
The core idea of (hard) SVM is to find a hyperplane which separates the training set
\textit{with the largest margin}. That is, we hope to solve the following:
\begin{equation}\label{SVM_original}
    \argmax_{\V{w}, b}\min_i \frac{|\T{\V{w}}\V{x}_i + b|}{||\V{w}||}
\end{equation}
Using the assumption that the hyperplane should correctly separates the training set, let
$$\gamma_i = \frac{y_i(\langle\V{w}, \V{x}_i\rangle + b)}{||\V{w}||}$$
and $\gamma =\displaystyle \min_i \gamma_i$~, then the problem~\eqref{SVM_original} becomes:
\begin{equation}
    \argmax_{\V{w}, b} \gamma\;,\quad \st \frac{y_i(\langle\V{w}, \V{x}_i\rangle + b)}{||\V{w}||} \geq \gamma
    \quad\forall\ i
\end{equation}
Let $\gamma \gets ||\V{w}||\gamma$, the above problem becomes:
\begin{equation}
    \argmax_{\V{w}, b} \frac{\gamma}{||\V{w}||}\;,\quad \st y_i(\langle\V{w}, \V{x}_i\rangle + b) \geq \gamma
    \quad\forall\ i
\end{equation}
Using the linear separable assumption again, we know that $\gamma > 0$. Let $\V{w} \gets \frac{\V{w}}{\gamma}$
and $b \gets \frac{b}{\gamma}$, then the above becomes:
\begin{equation}\label{SVM_argmax}
    \argmax_{\V{w}, b} \frac{1}{||\V{w}||}\;,\quad \st y_i(\langle\V{w}, \V{x}_i\rangle + b) \geq 1
    \quad\forall\ i
\end{equation}
which is obviously equivalent to:
\begin{equation}\label{hard_SVM}
    \argmin_{\V{w}, b} \frac{||\V{w}||^2}{2}\;,\quad \st y_i(\langle\V{w}, \V{x}_i\rangle + b) \geq 1
    \quad\forall\ i
\end{equation}

% validity of the above analysis.
\begin{thm}
    Assume the training set \dataset\ is \magenta{linear separable}, then there exists a unique hyperplane separating
    the dataset with the largest margin, which is given by the solution of the above 
    problem~\eqref{hard_SVM}.
\end{thm}
\begin{pf}
    The existence part is immediately from the separability assumption.\\
    TODO
\end{pf}

\begin{re}
    From the problem~\eqref{SVM_argmax}, it is clear that if $(\V{w}, b)$ is the separating hyperplane, then
    $\exists~i\ \st y_i(\langle\V{w}, \V{x}_i\rangle + b) = 1$ and the (largest) margin is $\frac{1}{||\V{w}||}$.
    Those $\V{x}_i$\,s are called \textbf{supporting vectors} of the hyperplane.
\end{re}

\subsection{Soft SVM}
The hard SVM works well when the data set is linear separable, but it behaves poorly on the sets which are not
linear separable since all the restrictions cannot be satisfied at the same time. In order to adapt to the 
non-separable case, we can allow some points to break the restriction, and penalize them in the optimization
target. That is, we can consider the following problem:
\begin{equation}\label{soft_SVM_original}
    \argmin_{\V{w}, b} \frac{||\V{w}||^2}{2} + C\sum_{i=1}^{m}\max(0, 1 - y_i(\T{\V{w}}\V{x}_i +b))
\end{equation}
Here, if $y_i(\T{\V{w}}\V{x}_i +b) < 1$, we add the penalization $1 - y_i(\T{\V{w}}\V{x}_i +b)$ to the 
optimization target; otherwise, no penalization is added. The constant $C > 0$ is the weight that determines 
how much the penalization matters (or how much you can violate the restrictions). For example, if $C = 0$, 
then the penalization dosen't matter and you can violate all the restrictions; if $C = +\infty$, then the 
penalization matters most and you cannot violate any single restriction.\par
If we introduce the slack variabels $\xi_i = \max(0, 1 - y_i(\T{\V{w}}\V{x}_i +b))$, then
the problem~\eqref{soft_SVM_original} becomes:
\begin{equation}\label{soft_SVM}
    \argmin_{\V{w}, b} \frac{||\V{w}||^2}{2} + C\sum_{i=1}^{m}\xi_i\quad\st\left\{
    \begin{aligned}
    & y_i(\langle\V{w}, \V{x}_i\rangle + b) \geq 1 - \xi_i\\
    & \xi_i \geq 0 
    \end{aligned}\right.
    \quad\forall~i
\end{equation}
this is what we called the soft SVM\@.

\subsection{Duality}


\subsection{Kernel method}
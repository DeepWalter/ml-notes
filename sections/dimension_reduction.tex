\section{Dimension Reduction}

\subsection{Low-dimensional embedding}
Let the feature space $\mathcal{X} = \mathbb{R}^d$. Our goal is to find an embedding of all samples into a
low dimension space $\mathbb{R}^{d'}$, here $d' \leq d$. That is, we want to find a map $e: \mathbb{R}^d
\longrightarrow \mathbb{R}^{d'}$ \st for any samples $\V{x}_i, \V{x}_j$, we have 
$$\dist(e(\V{x}_i), e(\V{x}_j)) = \dist(\V{x}_i, \V{x}_j)$$
Thus ${\left\{e(\V{x}_i)\right\}}_i$ is a low-dimensional embedding of the original samples. 

Let $\V{D} = {\left(\dist(\V{x}_i, \V{x}_j)\right)}_{m \times m}$ be the distance matrix of the samples 
${\left\{\V{x}_i\right\}}_i$, we want to find the embedding matrix $\V{Z} \in \mathbb{R}^{m \times d'}$, \st 
$||\V{z}_i - \V{z}_j|| = \V{D}_{ij}$, where $\V{Z} = (\V{z}_1; \dotsc; \V{z}_m)$ and 
$\V{D}_{ij} = \dist(\V{x}_i, \V{x}_j)$. Hence we have 
$$ \V{D}_{ij}^2 = ||\V{z}_i||^2 + ||\V{z}_j||^2 - 2 \langle\V{z}_i, \V{z}_j\rangle$$
Let $\V{B} = {\left(b_{ij}\right)}_{m \times m}= \V{Z}\T{\V{Z}}$ where $b_{ij} = \langle\V{z}_i, \V{z}_j\rangle$, 
we have
\begin{equation}\label{DR_Dij}
    \V{D}_{ij}^2 = b_{ii} + b_{jj} - 2 b_{ij}
\end{equation}
Let $\V{c} = \frac{1}{m}\sum_i \V{z}_i$, then ${\left\{\V{z}_i - \V{c}\right\}}_i$ is an embedding with the
property that $\sum_i(\V{z}_i - \V{c}) = 0$. Hence we can require that $\sum_i \V{z}_i = 0$ in the above
discussion. Then from equation~\eqref{DR_Dij}, we know that:
\begin{align}
    \sum_i \V{D}_{ij}^2 &= \tr(\V{B}) + m b_{jj}\\
    \sum_j \V{D}_{ij}^2 &= \tr(\V{B}) + m b_{ii}\\
    \sum_{i,j} \V{D}_{ij}^2 &= 2m \tr(\V{B})
\end{align}
From the above equations, we have
\begin{equation}\label{DR_bij}
    b_{ij} = -\frac{1}{2}\left(\V{D}_{ij}^2 - \frac{1}{m}\sum_i \V{D}_{ij}^2 - \frac{1}{m} \sum_j \V{D}_{ij}^2
    + \frac{1}{m^2}\sum_{i,j}\V{D}_{ij}^2\right)
\end{equation}
That is, $\V{B}$ is totally determined by $\V{D}$. Let $\V{B} = \V{V}\V{\Lambda}\T{\V{V}}$ be the eigenvalue
decomposition of $\V{B}$. Since $\V{B}$ is semi-positive, $\V{\Lambda}$ is a diagonal matrix with non-negative
diagonals. Let $\V{\Lambda}_*$ be the diagonal matrix obtained by removing the zero eigenvalues from 
$\V{\Lambda}$ and $\V{V}_*$ the matrix by removing the corresponding columns of $\V{V}$. Then it's easy to 
conclude that 
$$\V{Z} = \V{V}_*\V{\Lambda}_*^{\nicefrac{1}{2}}$$
is what we want. Hence the dimension $d'$ is totally determined by the distance matrix $\V{D}$. 

In practise, we often fix some $d' \ll d$ at first, then obtain $\V{\Lambda}_*$ by keeping the $d'$ largest 
eigenvalues and remove the rest, and $\V{V}_*$ the corresponding matrix. By this way, we may lose some 
precision in keeping the pairwise distance, but we can greatly reduce the dimension.

\begin{algorithm}
    \caption{Multiple Dimensional Scaling}
    \begin{algorithmic}[1]
        \Require the distance matrix $\V{D}_{m \times m}$; dimension $d'$.
        \State Compute the matrix $\V{B}$ according to equation~\eqref{DR_bij}.
        \State Eigenvalue decomposition: $\V{B} = \V{V} \V{\Lambda}\T{\V{V}}$.
        \State Let $\V{\Lambda}_*$ be the diagonal matrix with the $d'$ largest eigenvalues and $\V{V}_*$ the
        matrix with the corresponding eigenvectors.
        \Ensure Low-dimensional embedding: $\V{Z} = {\left(\V{V}_* \V{\Lambda}_*^{\nicefrac{1}{2}}\right)}_{m 
        \times d'}$
    \end{algorithmic}
\end{algorithm}

\subsection{Linear Dimension Reduction}
The simplest way to reduce dimension is by dropping some coordinates, that is, via projection. This keeps the
linear structure of the samples. A more generalized way is via linear transformation: $A: \mathbb{R}^d 
\longrightarrow \mathbb{R}^{d'}$. In matrix form, we want to find a transformation matrix 
$\V{W} \in \mathbb{R}^{d \times d'}$ \st for any row vector $\V{x} \in \mathbb{R}^d$, we have
$A(\V{x}) = \V{x} \V{W}$. If we write $\V{W} = \left(\V{w}_1, \dotsc, \V{w}_{d'}\right)$, then we have
$A(\V{x}) = \left(\langle \V{x}, \V{w}_1\rangle, \dotsc, \langle \V{x}, \V{w}_{d'}\rangle\right)$. If
${\left\{\V{w}_i\right\}}_{i=1}^{d'}$ is orthonormal, then $A(\V{x})$ is the projection of $\V{x}$ into the
subspace spanned by ${\left\{\V{w}_i\right\}}_{i=1}^{d'}$. It's easy to see that if 
$\V{X} = {\left(\V{x}_1; \dotsc; \V{x}_m\right)}_{m \times d}$ is the feature matrix of the samples, then 
$\V{Z} = \V{X} \V{W} \in \mathbb{R}^{m \times d'}$ is the representation matrix desired.

\subsubsection{Principle Component Analysis}
The naive projection method may fail because of the huge possibility of dropping those components that really
matters while keeping those that have little information about the dataset. To overcome this, we need to
determine what is it mean for a component to be principle. A natrual measurement is the variance. Namely, 
the component with largest variance and lowest covariance w.r.t.\ others counts most. We only drop out those 
with small variances. 
\par
Let $\V{X} = {\left(\V{x}_1; \dotsc; \V{x}_m\right)}$ be the feature matrix as above. Without loss of 
generality, we may assume that $\sum_i \V{x}_i = 0$. Then $\T{\V{X}}\V{X} \in \mathbb{R}^{d \times d}$ is the
covariance matrix of $\V{X}$. If $\T{\V{X}}\V{X}$ is diagonal, then the largest eigenvalue corresponds to the 
principle component. Hence we may keep the components corresponding to the $d'$ largest eigenvalues. If 
$\T{\V{X}}\V{X}$ is not diagonal, let $\T{\V{X}}\V{X} = \V{V}\V{\Lambda}\T{\V{V}}$ be its eigenvalue 
decomposition. Then we have $\T{\V{V}}\T{\V{X}}\V{X}\V{V} = \V{\Lambda}$. The matrix 
${\left(\V{V}_*\right)}_{d \times d'}$ consists of the eigenvectors corresponding to the $d'$ largest 
eigenvalues of $\V{\Lambda}$ is the desired transformation matrix. And $\V{X}\V{V}_*$ is the representation of
the original samples in low dimensional space $\mathbb{R}^{d'}$.

% algorithm of PCA
\begin{algorithm}
    \caption{Principle Component Analysis}\label{algorithm:pca}
    \begin{algorithmic}[1]
        \Require dataset $D = \left\{\V{x}_1, \dotsc, \V{x}_m\right\}$; dimension $d'$.
        \State Compute the center of samples: $\V{c} = \frac{1}{m}\sum_i \V{x}_i$
        \State $\V{x}_i \gets \V{x}_i -\V{c}$
        \State Compute the covariance matrix $\T{\V{X}}\V{X}$\Comment{$\V{X}=(\V{x}_1; \dotsc;\V{x}_m)$}
        \State Eigenvalue decomposition: $\T{\V{X}}\V{X} = \V{V}\V{\Lambda}\T{\V{V}}$
        \State Fetch the (column) eigenvectors $\V{w}_1, \dotsc, \V{w}_{d'}$ from $\V{V}$ corresponding to the
        $d'$ largest eigenvalues of $\V{\Lambda}$.
        \Ensure transformation matrix $\V{W} = \left(\V{w}_1, \dotsc, \V{w}_{d'}\right)$
    \end{algorithmic}
\end{algorithm}

\begin{re}
    The PCA method actually consists of two steps:
    \begin{compactenum}
        \item Centralization: $\V{x} \gets \V{x} -\V{c}\quad\forall~\V{x}$ where the center $\V{c}$ is computed
        from the the given data set.
        \item Linear transformation: $\V{z} = \V{x} \V{W}$ where $\V{z}, \V{x}$ are both row vectors and
        $\V{z}$ is the desired representation in low dimensional space.
    \end{compactenum}
    Of course, those two steps should also apply to unseen data.
\end{re}

\begin{re}
    In the above PCA algorithm~\ref{algorithm:pca}, we can use the \textbf{singular value decomposition}
    instead of the eigenvalue decomposition. Let $\V{X} = \V{U}\V{\Sigma}\T{\V{V}}$ be the singular value 
    decomposition of $\V{X}$, then we have $\T{\V{X}}\V{X} = \V{V} \T{\V{\Sigma}}\V{\Sigma} \T{\V{V}}$. Notice
    that $\T{\V{\Sigma}}\V{\Sigma}$ is a diagonal. We can proceed as in the above algorithm to get the
    transformation matrix.
\end{re}

There are other approaches to derive the PCA algorithm.
\section{Decision Trees}
Decision Tree.

% Pseudocode of decision tree algorithm.
\begin{algorithm}
    \caption{Decision Tree}\label{decision_tree}
    \begin{algorithmic}[1]
        \Require training set $D = \{(x_1, y_1), (x_2, y_2), \ldots, (x_m, y_m)
        \}$; attribute set $A = \{a_1, a_2, \ldots, a_d\}$.
        \Ensure a decision tree.
        \Procedure{DT}{$D, A$}
            \State Generate a node.
            \If{$y_i = c, \forall i = 1, \ldots, m$} \Comment{All samples have 
            the same label}
                \State mark the node as a leaf with label $c$; \Return
            \EndIf
            \If{$A = \emptyset$ \OR samples of $D$ take the same value on each
            attribute from $A$}\Comment{attributes from $A$ cannot distinguish
            elements of $D$}
                \State mark the node as a leaf with the majority label of $D$; 
                \Return
            \EndIf
            \State Choose the \textit{best} attribute $a_*$ from $A$.\label{measurement}
            \ForAll{possible value $a_*^v$ of $a_*$}
                \State generate a branch from node.
                \State let $D_v$ be all samples that take value $a_*^v$ on attribute
                $a_*$.
                \If{$D_v = \emptyset$} \Comment{no sample of $D$ takes value $a_*^v$
                on $a_*$}
                    \State mark the branch as a leaf with the majority label of 
                    $D$; \Return
                \Else
                    \State set the branch to \Call{DT}{$D_v, A-\{a_*\}$}.
                \EndIf
            \EndFor
        \EndProcedure
    \end{algorithmic}
\end{algorithm}


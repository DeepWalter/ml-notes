Let $X$ be the state space, $A$ the action space. Reinforcement learning can be described as a
\textbf{Markov Decision Process}: system changes from a state to another under the actions from $A$ with a 
rewarding function grading the change. That is, a reinforcement learning is a quadruple $E = \langle X, A, P,
R\rangle$, where $P: X \times A \times X \longrightarrow \mathbb{R}$ describes the probability of a state
changed into another via an action from $A$ and $R: X \times A \times X \longrightarrow \mathbb{R}$ describes
the reward of that change. Sometimes the rewarding function only depends on states, that is $R: X \times X
\longrightarrow \mathbb{R}$. Note that given a state $x$ and an action $a$ that can act on it, the resulting 
state $x'$ may not be unique, but the identity $\sum_{x'\in X}P_{x\rightarrow x'}^a = 1$ always holds true.
\par
The goal of reinforcement leanring is to find a \hl{policy $\pi$} dictating which action to take given a state
$x$. $\pi$ can be described in a determinate form $\pi: X \longrightarrow A$, which dictates that action
$\pi(x)$ must be performed on state $x$. It can also be described in a probability form: $\pi: X \times A
\longrightarrow \mathbb{R}$ where $\pi(x, a)$ is the probability of performing action $a$ on $x$. Obviously,
the determinate form is a special case of the probability form. The performance measurement of the policy is 
the accumulated reward gained from performing this policy for a reasonably long time. The \textbf{$T$ steps
accumulated reward} $\E[\frac{1}{T}\sum_{t=1}^{T}r_t]$ and \textbf{$\gamma$ discount accumulated reward}
$\E[\sum_{t=0}^{\infty}\gamma^t r_{t+1}]$ are two commonly used accumulated rewards, where $r_t$ is the reward
of step $t$.

\section{K-armed Bandit}
The K-armed bandit is a model of one step reinforcement learning. Every time you pull one arm of the machine,
it gives you some rewards with certain probability. The goal is to acquire as many rewards as possible.

\subsection{Exploration and Exploitation}
The \textbf{exploration-only} method tries to figure out the expectation reward of each arm. It divides the 
exploration chance evenly on each arm, and use the average reward of each arm as the expectation reward. This
method may do a good job in estimating each arm's reward, but it may miss the opportunity to get the most 
rewards. The \textbf{exploitation-only} method, on the other hand, only pulls the arm that gives the most 
average rewards so far. Since it dosen't care about the expectation reward of each arm, it may also miss the
opportunity to get the most rewards. 

\subsection{$\varepsilon$-greedy}
The $\varepsilon$-greedy is a compromise between exploration-only and exploitation-only method. At
each try, it performs exploration-only method with a probability $\varepsilon$ and exploitation-only method
the other $1 - \varepsilon$.\par
Let $Q_n(k)$ denote the average reward of arm $k$ with $n$ tries and $v_n^k$ its n-th reward, then it is 
clearly that
$$Q_n(k) = \frac{1}{n} ((n-1)Q_{n-1}(k) + v_n^k)$$
This is the updating rules for the average reward.

% epsilon-greedy algorithm
\begin{algorithm}
    \caption{$\varepsilon$-greedy}\label{epsilon_greedy_for_K_arm_bandit}
    \begin{algorithmic}[1]
        \Require number of arms $K$; rewarding function $R$; number of tries $T$; exploration threshold 
        $\varepsilon$.
        \State $r = 0$\Comment{the accumulated reward.}
        \State $Q(i) = 0, \quad count(i) = 0 \quad \forall~i=1,\ldots,K$\Comment{Initialization.}
        \For{$t = 1$ \algorithmicto $T$}
            \If{$rand() < \varepsilon$}
                \State $k = rand(\{1, \ldots, K\})$\Comment{choose with equal probability}
            \Else
                \State $k = \argmax_i Q(i)$
            \EndIf
            \State $v = R(k)$
            \State $r \gets r + v$
            \State $Q(k) \gets \frac{count(k) \cdot Q(k) + v}{count(k) + 1}$
            \State $count(k) \gets count(k) + 1$
        \EndFor
        \Ensure the accumulated reward $r$.
    \end{algorithmic}
\end{algorithm}

% TODO: analysis of the epsilon-greedy.

\subsection{Softmax}
Softmax algorithm is another way of compromising between the exploration-only and exploitation-only methods. 
Unlike the $\varepsilon$-greedy algorithm which uses a threshold to determine how to choose an arm, softmax 
attach each arm with a probability to be chosen base on its current average reward. The probability 
distribution among the arms is a Boltzmann distribution, namely:
\begin{equation}\label{K_arm_bandit_softmax}
P(k) = \frac{e^{\frac{Q(k)}{\tau}}}{\sum_{i=1}^K e^{\frac{Q(i)}{\tau}}}\quad\forall~k = 1,\ldots,K
\end{equation}
where the parameter $\tau > 0$. Apparently, if $\tau$ is close to $0$, the softmax favours the arm with
the highest average reward, which is the case of exploitation-only method; if $\tau$ is very large (close to 
$+\infty$), the softmax degenerates to uniform distribution, which is the case of exploration-only method.

% softmax algorithm for K-arm bandit.
\begin{algorithm}
    \caption{Softmax}\label{Softmax_for_K_arm_bandit}
    \begin{algorithmic}[1]
        \Require number of arms $K$; rewarding function $R$; number of tries $T$; parameter $\tau$.
        \State $r = 0$
        \State $Q(i) = 0, \quad count(i) = 0\quad\forall~i=1,\ldots, K$
        \For{$t = 1$ \algorithmicto $T$}
            \State choose $k$ according to the distribution given by equation~\eqref{K_arm_bandit_softmax}.
            \State $v = R(k)$
            \State $r \gets r + v$
            \State $Q(k) \gets \frac{count(k)\cdot Q(k)}{count(k) + 1}$
            \State $count(k) \gets count(k) + 1$
        \EndFor
        \Ensure the accumulated reward $r$.
    \end{algorithmic}
\end{algorithm}

% TODO: analysis of the softmax algorithm.

\newpage
\section{Model-based Learning}
In model-based learning, the quadruple $E = \langle X, A, P, R\rangle$ are known to us. That is, we know all
the possible states, all the possible actions that may act on them, the transition function which describes
the probability of one state changing into another under some action, the rewarding function which grades the
change. In the following discussion, we assume that \magenta{the state space $X$ and the action space $A$ are
finite}.

\subsection{Policy Measurement}
Let $V^\pi(x)$ be the expected reward gained from applying policy $\pi$ on the starting state $x$, 
$Q^\pi(x, a)$ the expected reward gained from first applying action $a$ then policy $\pi$ on the starting
state $x$. $V$ and $Q$ are called \textbf{state value function} and \textbf{state-action value function}
respectively. \par
By definition, we can write the state value function as:
\begin{align}
    V^\pi_T(x) &= \E_\pi\left[\frac{1}{T}\sum_{t=1}^{T}r_t | x_0 = x\right], &\text{$T$ steps}
    \label{state_value_T}\\
    V^\pi_\gamma(x) &= \E_\pi\left[\sum_{t=0}^{\infty} \gamma^t r_{t+1} | x_0 = x\right], &\text{$\gamma$ 
    discount}\label{state_value_gamma}
\end{align}
Similarly, we can write the state-action value function as:
\begin{align}
    Q^\pi_T(x, a) &= \E_\pi\left[\frac{1}{T}\sum_{t=1}^T r_t | x_0 = x, a_0 = a\right], &\text{$T$ steps}
    \label{state-action_value_T}\\
    Q^\pi_\gamma(x, a) &= \E_\pi\left[\sum_{t=0}^\infty \gamma^t r_{t+1} | x_0 = x, a_0 = a\right] &\text{
    $\gamma$ discount}\label{state-action_value_gamma}
\end{align}
\par
Since the next state of the system only depends on the current state, those value functions can be written in
a recursive form. For example, the $T$ steps accumulated state value function~\eqref{state_value_T} can be 
written as:
\begin{align}
    V^\pi_T(x) &= \E_\pi \left[\frac{1}{T}\sum_{t=1}^T r_t | x_0 = x\right]\nonumber\\
               &= \E_\pi \left[\frac{1}{T} r_1 + \frac{T-1}{T}\frac{1}{T-1}\sum_{t=2}^T r_t | x_0 = x\right]
               \nonumber\\
               &= \sum_{a\in A}\pi(x, a)\sum_{x'\in X}P_{x\rightarrow x'}^a\left(\frac{1}{T}
               R_{x\rightarrow x'}^a + \frac{T-1}{T}\E_\pi\left[\frac{1}{T-1}\sum_{t=1}^{T-1}r_t | x_0=x'
               \right]\right)\nonumber\\
               &= \sum_{a\in A}\pi(x, a)\sum_{x'\in X}P_{x\rightarrow x'}^a\left(\frac{1}{T}
               R_{x\rightarrow x'}^a + \frac{T-1}{T} V^\pi_{T-1}(x')\right)\label{state_value_T_expansion}
\end{align}

% algorithm for calculating T steps accumulated state value function.
\begin{algorithm}
    \caption{$T$ steps accumulated state value function}
    \begin{algorithmic}[1]
        \Require $E = \langle X, A, P, R\rangle$; policy $\pi$; steps $T$.
        \State $V(x) = 0\quad\forall~x\in X$
        \For{$t=1$ \algorithmicto $T$}
            \State $\displaystyle V'(x) = \sum_{a\in A}\pi(x,a)\sum_{x'\in X}P_{x\rightarrow x'}^a 
            \left(\frac{1}{t} R_{x\rightarrow x'}^a + \frac{t-1}{t}V(x')\right)\quad\forall~x\in X$
            \If{$t = T+1$}
                \State \algorithmicbreak
            \Else
                \State $V \gets V'$
            \EndIf
        \EndFor
        \Ensure state value function $V$.
    \end{algorithmic}
\end{algorithm}

Similarly, the $\gamma$ discount accumulated state value function~\eqref{state_value_gamma} can be written as:
\begin{align}
    V^\pi_\gamma(x) &= \E_\pi\left[\sum_{t=0}^\infty \gamma^t r_{t+1} | x_0 = x\right]\nonumber\\
                    &= \E_\pi\left[r_0 + \gamma \sum_{t=1}^\infty \gamma^{t-1}r_{t+1} | x_0=x\right]\nonumber\\
                    &= \sum_{a\in A}\pi(x, a)\sum_{x'\in X}P_{x\rightarrow x'}^a \left(R_{x\rightarrow x'}^a + 
                    \gamma \E_\pi\left[\sum_{t=0}^\infty \gamma^t r_{t+1} | x_0 = x'\right]\right)\nonumber\\
                    &= \sum_{a\in A}\pi(x, a)\sum_{x'\in X}P_{x\rightarrow x'}^a \left(R_{x\rightarrow x'}^a +
                    \gamma V^\pi_\gamma(x')\right)\label{state_value_gamma_expansion}
\end{align}

% algorithm for gamma discount accumulated state value function.
\begin{algorithm}
    \caption{$\gamma$ discount accumulated state value function}
    \begin{algorithmic}[1]
        \Require $E = \langle X, A, P, R\rangle$; policy $\pi$; parameter $\gamma$; threshold $\theta$.
        \State $V(x) = 0\quad\forall~x\in X$
        \Loop
            \State $\displaystyle V'(x) = \sum_{a\in A}\pi(x,a)\sum_{x'\in X}P_{x\rightarrow x'}^a 
            \left(R_{x\rightarrow x'}^a + \gamma V(x')\right)\quad\forall~x\in X$
            \If{$\displaystyle\max_{x\in X}|V(x) - V'(x)| < \theta$}
                \State \algorithmicbreak
            \Else
                \State $V \gets V'$
            \EndIf
        \EndLoop
        \Ensure state value function $V$.
    \end{algorithmic}
\end{algorithm}

% formulas for state-action value functions.
Since the state-action value functions are just the state value functions with the given first action, we can 
easily write the state-action value functions in terms of state value functions:
\begin{align}
    Q^\pi_T(x, a) &= \sum_{x' \in X} P_{x \rightarrow x'}^a\left(\frac{1}{T}R_{x \rightarrow x'}^a +
    \frac{T-1}{T}V^\pi_{T-1}(x')\right)\label{state-action_value_T_expansion}\\
    Q^\pi_\gamma(x, a) &= \sum_{x' \in X} P_{x \rightarrow x'}^a\left(R_{x \rightarrow x'}^a + \gamma 
    V^\pi_\gamma(x')\right)\label{state-action_value_gamma_expansion}
\end{align}
Hence it is obvious that:
\begin{equation}\label{relation_V_Q}
V^\pi(x) = \sum_{a \in A}\pi(x, a)Q^\pi(x, a)
\end{equation}

% TODO: algorithms for state-action value functions.

\subsection{Policy Improvement}
Of course, the best policy is the one that maximize the state value function, i.e.
$$\pi^* = \argmax_\pi \sum_{x \in X}V(x)$$
Due to the recursive nature of the state value functions~\eqref{state_value_T_expansion}
and~\eqref{state_value_gamma_expansion}, we can conclude that if there is no restriction on the policies that 
we can choose, the optimal state value functions satisfy the following equations:
\begin{align}
    V^*_T(x) &= \max_{a \in A} \sum_{x' \in X}P_{x \rightarrow x'}^a\left(\frac{1}{T}R_{x \rightarrow x'}^a +
    \frac{T-1}{T}V^*_{T-1}(x')\right)\\
    V^*_\gamma(x) &= \max_{a \in A} \sum_{x' \in X}P_{x \rightarrow x'}^a\left(R_{x \rightarrow x'}^a + \gamma
    V^*_\gamma(x')\right)
\end{align}
That is,
\begin{equation}\label{relation_optimal_V_Q}
    V^*(x) = \max_{a \in A}Q^*(x, a)
\end{equation}
This in turn gives us the expression of $Q^*(x, a)$:
\begin{align}
    Q^*_T(x, a) &= \sum_{x' \in X} P_{x \rightarrow x'}^a\left(\frac{1}{T}R_{x \rightarrow x'}^a +
    \frac{T-1}{T}\max_{a' \in A}Q^*_{T-1}(x',a')\right)\\
    Q^*_\gamma(x, a) &= \sum_{x' \in X} P_{x \rightarrow x'}^a\left(R_{x \rightarrow x'}^a + \gamma 
    \max_{a' \in A}Q^*_\gamma(x', a')\right)
\end{align}

The relation between optimal state value function and optimal state-action value 
function~\eqref{relation_optimal_V_Q} shed light on how to improve the current policy: \magenta{always choose 
the current best action for the state}. That is, let 
\begin{equation}
\pi'(x) = \argmax_{a \in A}Q^\pi(x, a)\quad\forall~x \in X
\end{equation}
Then by relation~\eqref{relation_V_Q}, we have
$$Q^\pi(x, \pi'(x)) \geq V^\pi(x)$$
Since $\pi'(x)$ is the action that should be taken on the state $x$, by relation~\eqref{relation_V_Q} again, 
we have
\begin{align*}
    V^{\pi'}(x) &= \sum_{a \in A}\pi'(x, a) Q^{\pi'}(x, a)\\
              &= Q^{\pi'}(x, \pi'(x))
\end{align*}
This gives us the following estimate:
\begin{align*}
    V^\pi(x) - V^{\pi'}(x) &\leq Q^\pi(x, \pi'(x)) - Q^{\pi'}(x, \pi'(x))\\
                           &\leq \sum_{x' \in X}P_{x \rightarrow x'}^{\pi'(x)}\cdot\gamma\left(V^\pi(x') - 
                           V^{\pi'}(x')\right)\\
                           &\leq \gamma \max_{x' \in X}\left(V^\pi(x') - V^{\pi'}(x')\right)\sum_{x' \in X}
                           P_{x \rightarrow x'}^{\pi'(x)}\\
                           &\leq \gamma \max_{x' \in X}\left(V^\pi(x') - V^{\pi'}(x')\right)
\end{align*}
Here, we use the fact that $P \geq 0$ and $\sum_{x' \in X}P_{x \rightarrow x'}^{\pi'(x)} = 1$. Since
$\gamma \in (0, 1)$, we conclude that
$$\max_{x \in X}\left(V^\pi(x) - V^{\pi'}(x)\right) \leq 0$$
That is 
$$V^\pi(x) \leq V^{\pi'}(x)\quad\forall~x \in X$$
This justifies the choice of $\pi'(x)$.
\begin{re}
    Note that the above discussion is also valid in the case of $T$-steps accumulated state value function
    with almost the same argument.
\end{re}

% TODO: policy iteration

% TODO: value iteration

\section{Model-free Learning}
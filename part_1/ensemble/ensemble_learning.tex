\section{AdaBoost}
AdaBoost combines a family of binary classification base (weak) learners from an algorithm linearly to get a stronger (more
powerful) one. It takes advantage
of a former learner and use its output to force the latter learner to focus more on those data on which the
former learner performs poorly. In this way, it hopes to make the aggregate learner performs well on all
training data. Thus AdaBoost focus more on reducing the bias. 
% AdaBoost algorithm
\begin{algorithm}
    \caption{AdaBoost}\label{AdaBoost}
    \begin{algorithmic}[1]
        \Require training set $D = \{(\V{x}_1, y_1), \ldots, (\V{x}_m, y_m)\}$; base learner $\mathcal{L}$;
        training round $T$.
        \State $\mathcal{D}_1(\V{x}) = \frac{1}{m}$.\Comment{$\mathcal{D}$ is a distribution over the input 
        examples.}
        \For{$t = 1$ \algorithmicto $T$}
            \State $h_t = \mathcal{L}(D, \mathcal{D}_t)$;
            \State $\varepsilon_t = \p_{\V{x}\sim\mathcal{D}_t}(h_t(\V{x})\neq f(\V{x}))$;\Comment{
            $\varepsilon_t$ is the error rate of $h_t$ w.r.t $\mathcal{D}_t$.}
            \breakif{$\varepsilon_t > 0.5$}\Comment{Base learner must be above average.}
            \State $\alpha_t = \frac{1}{2}\ln{\frac{1 - \varepsilon_t}{\varepsilon_t}}$;\label{AdaBoost_alphat}
            \State $\mathcal{D}_{t + 1}(\V{x}) = \frac{\mathcal{D}_t(\V{x})\exp(-\alpha_t f(\V{x})h_t(\V{x}))}
            {Z_t}$.\Comment{$Z_t$ is the normalization constant.}\label{AdaBoost_distritution}
        \EndFor
        \Ensure $H(\V{x}) = \sign(\displaystyle\sum_{t = 1}^T \alpha_t h_t(\V{x}))$
    \end{algorithmic}
\end{algorithm}

%TODO: equivalent to 0/1 loss function.
Consider the exponential loss function
\begin{equation}
    L_{\exp}(H | \mathcal{D}) = \E_{\V{x}\sim\mathcal{D}}[e^{-f(\V{x})H(\V{x})}]
\end{equation}
We want to find a linear combination of base learners 
$$ H(\V{x}) = \sum_{t=1}^T \alpha_t h_t(\V{x})$$
that minimize it. Although it is difficult to find such $H$ as a whole, we can approach this problem step by
step. That is, we first find some $h_1$ and $\alpha_1$ \st $H_1 = \alpha_1 h_1$ minimize the exponential loss,
then we find some $h_2$ and $\alpha_2$ \st $H_2 = H_1 + \alpha_2 h_2$ minimize the exponential loss too. And
we carray on this process until $t = T$. Assume we have $H_{t-1}$ and want to find the next learner and its
weight, that is, we want to find $\alpha_t$ and $h_t$ \st $H_t = H_{t-1} + \alpha_t h_t$ minimize the 
exponential loss. Since
\begin{align*}
    L_{\exp}(H_t | \mathcal{D}) &= L_{\exp}(H_{t-1} + \alpha_t h_t | \mathcal{D})\\
    &= \E_{\V{x}\sim \mathcal{D}}[e^{-f(\V{x})H_{t-1}(\V{x})}\cdot e^{-f(\V{x})\alpha_t h_t(\V{x})}]\\
    &= C_{t-1} \E_{\V{x}\sim \mathcal{D}}[\frac{e^{-f(\V{x})H_{t-1}(\V{x})}}{C_{t-1}}
    e^{-f(\V{x})\alpha_t h_t(\V{x})}]\\
    &= C_{t-1} \E_{\V{x}\sim \mathcal{D}_t}[e^{-f(\V{x})\alpha_t h_t(\V{x})}]
\end{align*}
where $C_{t-1} = \E_{\V{x}\sim\mathcal{D}}[e^{-f(\V{x})H_{t-1}(\V{x})}]$ is a constant and 
$\mathcal{D}_t = \frac{e^{-f(\V{x})H_{t-1}(\V{x})}}{C_{t-1}} \mathcal{D}$ is a distribution, we have
\begin{align}
    L_{\exp}(H_t | \mathcal{D}) &= C_{t-1} \left\{e^{-\alpha_t} \p_{\V{x}\sim\mathcal{D}_t}(f(\V{x})=h_t(\V{x})) 
    + e^{\alpha_t}\p_{\V{x}\sim\mathcal{D}_t}(f(\V{x})\neq h_t(\V{x}))\right\}\nonumber\\
    &= C_{t-1} \{e^{-\alpha_t}(1 - \varepsilon_t) + e^{\alpha_t}\varepsilon_t\}\label{Ada_exp_loss_expansion}
\end{align}
where $\varepsilon_t$ is the error rate of $h_t$ w.r.t.\ the distribution $\mathcal{D}_t$. Hence 
$$\pfrac{ }{\alpha_t}L_{\exp}(H_t | \mathcal{D}) = 0$$
implies 
\begin{equation}\label{AdaBoost_weight}
    \alpha_t = \frac{1}{2}\ln(\frac{1-\varepsilon_t}{\varepsilon_t})
\end{equation}
which justifies the choice of $\alpha_t$ in line~\algref{AdaBoost}{AdaBoost_alphat}. Moreover, 
$$\mathcal{D}_t = \frac{e^{-f(\V{x})H_{t-1}(\V{x})}}{C_{t-1}} \mathcal{D}$$
implies
\begin{align}
\mathcal{D}_{t+1}(\V{x}) &= \frac{e^{-f(\V{x})\alpha_t h_t(\V{x})}C_{t-1}\mathcal{D}_t(\V{x})}{C_t}\nonumber\\
                         &= \frac{e^{-f(\V{x})\alpha_t h_t(\V{x})}\mathcal{D}_t(\V{x})}{Z_t}
\end{align}
which justifies the choice of $\mathcal{D}_t$ in line~\algref{AdaBoost}{AdaBoost_distritution}. If we plug
equation~\eqref{AdaBoost_weight} into~\eqref{Ada_exp_loss_expansion}, we get
$$L_{\exp}(H_t | \mathcal{D}) = 2C_{t-1}\cdot\sqrt{\varepsilon_t (1-\varepsilon_t)}$$
Since the performance of the base learner is slightly above the average, that is, 
$\varepsilon_t < \frac{1}{2}$, the minimum of the exponential loss is achieved at the $h_t$ which has 
the minimal error rate w.r.t.\ $\mathcal{D}_t$. That is, $h_t$ is the output of the algorithm w.r.t.\ the 
distribution $\mathcal{D}_t$ over the training data.


\section{Bagging}
Bagging is short for bootstrap aggregating, namely we use bootstrap sampling to sample $T$ groups of training
data, and train $T$ base learners using those groups, then we aggregate those $T$ base learners into one via 
plurality vote (for classification) or averaging (for regression). In this way, bagging mainly focus on
reducing the variance.

% Bagging algorithm
\begin{algorithm}
    \caption{Bagging}\label{Bagging}
    \begin{algorithmic}[1]
        \Require training set $D = \{(\V{x}_1, y_1), \dotsc, (\V{x}_m, y_m)\}$; base learner $\mathcal{L}$;
        training round $T$.
        \For{$t = 1$ \algorithmicto $T$}
            \State $h_t = \mathcal{L}(D, \mathcal{D}_{bs})$\Comment{bs stands for bootstrap sampling.} 
        \EndFor
        \Ensure $H(\V{x}) = \displaystyle\argmax_{y\in \mathcal{Y}}\sum_{t=1}^T 1(h_t(\V{x}) = y)$\Comment{$H$
        is the plurality vote of $h_t$.}
    \end{algorithmic}
\end{algorithm}

Unlike the AdaBoost in the preceding section, bagging can be applied to multiclass classification and
regression without any modifications. Morover, due to bootstrap sampling, we have around $36.8\%$ samples
untouched in each training, which can be used to do the out-of-bag estimation for testing the generalization
ability.

\section{Random Forest}
A forest is a collection of trees. When applying bagging to decision tree, we get a forest. A random forest is
obtained this way with randomness. Namely, when deciding the best feature for an internal node, we restrict 
the candidates to an randomly chosen subset from all possible features. Usually, the size of subset is 
$\log_2 d$ where $d$ is the total number of all possible features.

% 